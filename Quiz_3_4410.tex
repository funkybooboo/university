\documentclass[10pt, AMS Euler]{article}
\textheight=9.25in \textwidth=7in \topmargin=-.75in
\oddsidemargin=-0.25in
\evensidemargin=-0.25in
\usepackage{url}  % The bib file uses this
\usepackage{graphicx} %to import pictures
\usepackage{amsmath, amssymb, color, wasysym}
\usepackage{theorem, concrete, multicol,tikz}
\usepackage[normalem]{ulem} %for strikethrough (\sout{blah})
\usepackage{enumitem}

\setlength{\intextsep}{5mm} \setlength{\textfloatsep}{5mm}
\setlength{\floatsep}{5mm}


{\theorembodyfont{\rmfamily}
	\newtheorem{definition}{Definition}[section]}
{\theorembodyfont{\rmfamily} \newtheorem{example}{Example}[section]}
{\theorembodyfont{\rmfamily} \newtheorem{lemma}{Lemma}[section]}
{\theorembodyfont{\rmfamily} \newtheorem{theorem}{Theorem}[section]}
{\theorembodyfont{\rmfamily} \newenvironment{proof}{\par{\it
			Proof:}}{\nopagebreak[4]\rule{2mm}{2mm}}}
{\theorembodyfont{\rmfamily}
	\newenvironment{solution}{\par{\bf{Solution:}}}{\nopagebreak[4]\rule{2mm}{2mm}}}

\usetikzlibrary{arrows}
\usetikzlibrary{shapes}
\newcommand{\mymk}[1]{%
	\tikz[baseline=(char.base)]\node[anchor=south west, draw,rectangle, rounded corners, inner sep=2pt, minimum size=7mm,
	text height=2mm](char){\ensuremath{#1}} ;}

\newcommand*\circled[1]{\tikz[baseline=(char.base)]{
		\node[shape=circle,draw,inner sep=2pt] (char) {#1};}}


%%%%  SHORTCUT COMMANDS  %%%%
\newcommand{\ds}{\displaystyle}
\newcommand{\Z}{\mathbb{Z}}
\newcommand{\arc}{\rightarrow}
\newcommand{\R}{\mathbb{R}}
\newcommand{\N}{\mathbb{N}}
\newcommand{\Q}{\mathbb{Q}}
\newcommand{\stirling}[2]{\genfrac{\{}{\}}{0pt}{}{#1}{#2}}

%%%%  footnote style %%%%

\renewcommand{\thefootnote}{\fnsymbol{footnote}}

\pagestyle{empty}
\begin{document}
	
	\noindent{\bf \large MATH 4410 }\\
        Nate Stott A02386053
	
	\noindent \underline{\hspace{2in}}\\
	
	{\bf Quiz \#3; Due 11:59 pm, 1/24/2024}\\
        

        \newpage
	\begin{enumerate}
		
		\item Find a closed formula for this recurrence:
		\begin{align*} C_0 &= 0;\\ C_n &= n+1 + \frac{2}{n}\sum_{k=0}^{n-1}C_k, \;\;\;\; \mbox{for $n>0$.}\end{align*}
        \end{enumerate}

            Given
                $$ a_n T_n = b_n T_{n-1} + c_n $$
                $$ T_n = \frac{1}{s_n a_n} (s_0 a_0 T_0 + \sum_{k=1}^n s_k c_k)$$

            (Got this idea from Katlyn in the class) Rewrite $C_n$ so that it's in the form, 
                $$ a_n C_n = b_n C_{n-1} + c_n $$
            \\ 
            \\
            Find $C_{n-1}$
                $$ C_{n-1} = (n-1) + 1 + (\frac{2}{(n-1)}) \sum_{k=0}^{(n-1)-1} C_k $$
                $$ = n + (\frac{2}{n-1}) \sum_{k=0}^{n-2} C_k $$
            Ok, so I need to get $n + (\frac{2}{n-1}) \sum_{k=0}^{n-2} C_k$ in $C_n$
            \\ 
            \\
            \\
            \\
            Start
                $$ C_n = n + 1 + (\frac{2}{n}) \sum_{k=0}^{n-1} C_k $$
            Pull the $n-1$ iteration off the summation
                $$ = (\frac{2}{n}) C_{n-1} + 1 + n + (\frac{2}{n}) \sum_{k=0}^{n-2} C_k $$
            Change $\frac{2}{n}$ to $(\frac{n-1}{n})(\frac{2}{n-1})$
                $$ = (\frac{2}{n}) C_{n-1} + 1 + n + (\frac{n-1}{n})(\frac{2}{n-1}) \sum_{k=0}^{n-2} C_k $$
            Add $-n+n$ to $\frac{2}{n-1}$
                $$ = (\frac{2}{n}) C_{n-1} + 1 + n + (\frac{n-1}{n})( -n + n + \frac{2}{n-1}) \sum_{k=0}^{n-2} C_k $$
            Distribute
                $$ = (\frac{2}{n}) C_{n-1} + 1 + n + \frac{n-1}{n}(-n) + \frac{n-1}{n}(n) + \frac{n-1}{n}(\frac{2}{n-1}) \sum_{k=0}^{n-2} C_k $$
            Simplify
                $$ = (\frac{2}{n}) C_{n-1} + 1 + n - (n-1) + \frac{n-1}{n} (n + \frac{2}{n-1}) \sum_{k=0}^{n-2} C_k $$
            Bro, I can substitute $(n + \frac{2}{n-1}) \sum_{k=0}^{n-2} C_k$ for $C_{n-1}$
                $$ = (\frac{2}{n}) C_{n-1} + 1 + n - (n-1) + \frac{n-1}{n} C_{n-1} $$
            Simplify
                $$ = (\frac{2}{n}) C_{n-1} + \frac{n-1}{n} C_{n-1} + 2$$
                $$ = \frac{(2)C_{n-1}}{n}  + \frac{(n-1)C_{n-1}}{n} + 2$$
                $$ = \frac{(2)C_{n-1} + (n-1)C_{n-1}}{n} + 2$$
                $$ = C_{n-1}\frac{( 2 + n - 1)}{n} + 2$$
                $$ = C_{n-1}\frac{(n + 1)}{n} + 2$$
            Alright so 
                $$C_n = \frac{(n + 1)}{n} C_{n-1} + 2 $$
            \\
            \\
            \\
            So now I can get
                $$ a_n = 1 $$
                $$ b_n = \frac{(n + 1)}{n} $$
                $$ c_n = 2 $$
                $$ s_n = \frac{1}{\frac{(n+1)!}{n!}} = \frac{n!}{(n+1)!} = \frac{1}{n+1} $$
            Multiply by $s_n$
                $$ \frac{1}{n+1}C_n = \frac{1}{n+1} \frac{(n + 1)}{n} C_{n-1} + \frac{2}{n+1} $$
                $$ \frac{1}{n+1}C_n = \frac{1}{n} C_{n-1} + \frac{2}{n+1} $$
            Summation time
                $$ C_n = \frac{1}{\frac{1}{n+1}} [(1)(1)(0) + \sum_{k=1}^n( \frac{2}{k+1})]$$
                $$ = (n+1) (2)(\sum_{k=1}^n( \frac{1}{k+1} )$$
            Focus in on $\sum_{k=1}^n \frac{1}{k+1} $
                $$\sum_{k=1}^n \frac{1}{k+1} $$
                $$ = \sum_{k=1}^n k^{\underline{-1}}$$
                $$ = \left .H_k\right|_{k=1}^{k=n+1}$$
                $$ = H_{n+1} - H_1 $$
                $$ = H_{n+1} - 1 $$
            Alright back to where we came from
                $$ = (n+1)(2)(H_{n+1} - 1) $$
                $$ = 2nH_{n+1} - 2n + 2H_{n+1} - 2 $$
            \\
            \\
            Well there you have it
                $$ C_n = 2nH_{n+1} - 2n + 2H_{n+1} - 2 $$

            I checked with n = 1, 2, 3.
            

        \newpage
        \begin{enumerate}[resume]
		\item Find a closed formula for this recurrence:
		\begin{align*} T_0 &= 5;\\ 2T_n &= nT_{n-1} +3 \cdot n!, \;\;\;\; \mbox{for $n>0$.}\end{align*}
	\end{enumerate}
            Given
                $$ T_n = \frac{1}{s_n a_n} (s_0 a_0 T_0 + \sum_{k=1}^n s_k c_k)$$
	    Find
                $$ a_n = 2 $$
                $$ b_n = n $$
                $$ c_n = 3n! $$
                $$ S_n = \frac{2^{n-1}}{n!} $$
            Multiply
                $$ \frac{2^{n-1}}{n!} (2T_n) = \frac{2^{n-1}}{n!} (nT_{n-1} +3 \cdot n!) $$
                $$ \frac{2^n}{n!} T_n = \frac{2^{n-1}}{n!} nT_{n-1} + \frac{2^{n-1}}{n!} (3 \cdot n!) $$
                $$ \frac{2^n}{n!} T_n = \frac{2^{n-1}}{(n-1)!} T_{n-1} + 2^{n-1}(3)  $$
            Summation time
                $$ T_n = \frac{n!}{2^n} ((\frac{1}{2})(2)(5) + \sum_{k=1}^n 3(2^{k-1})) $$
                $$ = \frac{n!}{2^n} (5 + 3 \sum_{k=1}^n 2^{k-1} ) $$
            Focus in on $\sum_{k=1}^n 2^{k-1} $
                $$\sum_{k=1}^n 2^{k-1} = x = 2^0 + 2^1 + 2^2 + ... + 2^{n-1}$$
                $$ x = 1 + 2^1 + 2^2 + ... + 2^{n-1} $$
                $$ 2 x = 2 + 2^2 + 2^3 + ... + 2^n $$
                $$ 2 x - x = (2 + 2^2 + 2^3 + ... + 2^n) - (1 + 2^1 + 2^2 + ... + 2^{n-1}) $$
                $$ x = 2^n - 1 $$
            Back to where we came from 
                $$ = \frac{n!}{2^n} (5 + 3 (2^n - 1 )) $$
                $$ = \frac{n!}{2^n} (5 +  (3) 2^n - 3 ) $$
                $$ = \frac{n!}{2^n} ((3) 2^n + 2 ) $$
                $$ = (\frac{n!}{2^n})(3)(2^n) + (\frac{n!}{2^n})(2) $$
                $$ = (n!)(3) + \frac{(2)(n!)}{2^n} $$
                $$ = (3)(n!) + \frac{n!}{2^{n-1}} $$
            There you have it
                $$ T_n = (3)(n!) + \frac{n!}{2^{n-1}} $$

            Checked for n = 1, 2, 3
	
	\noindent \underline{\hspace{3in}}\\
	
	
	
	
\end{document}