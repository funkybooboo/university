\documentclass[10pt, AMS Euler]{article}
\textheight=9.25in \textwidth=7in \topmargin=-.75in
\oddsidemargin=-0.25in
\evensidemargin=-0.25in
\usepackage{url}  % The bib file uses this
\usepackage{graphicx} %to import pictures
\usepackage{amsmath, amssymb, color, wasysym}
\usepackage{theorem, concrete, multicol,tikz}
\usepackage[normalem]{ulem} %for strikethrough (\sout{blah})
\usepackage{enumitem}

\setlength{\intextsep}{5mm} \setlength{\textfloatsep}{5mm}
\setlength{\floatsep}{5mm}


{\theorembodyfont{\rmfamily}
	\newtheorem{definition}{Definition}[section]}
{\theorembodyfont{\rmfamily} \newtheorem{example}{Example}[section]}
{\theorembodyfont{\rmfamily} \newtheorem{lemma}{Lemma}[section]}
{\theorembodyfont{\rmfamily} \newtheorem{theorem}{Theorem}[section]}
{\theorembodyfont{\rmfamily} \newenvironment{proof}{\par{\it
			Proof:}}{\nopagebreak[4]\rule{2mm}{2mm}}}
{\theorembodyfont{\rmfamily}
	\newenvironment{solution}{\par{\bf{Solution:}}}{\nopagebreak[4]\rule{2mm}{2mm}}}

\usetikzlibrary{arrows}
\usetikzlibrary{shapes}
\newcommand{\mymk}[1]{%
	\tikz[baseline=(char.base)]\node[anchor=south west, draw,rectangle, rounded corners, inner sep=2pt, minimum size=7mm,
	text height=2mm](char){\ensuremath{#1}} ;}

\newcommand*\circled[1]{\tikz[baseline=(char.base)]{
		\node[shape=circle,draw,inner sep=2pt] (char) {#1};}}


%%%%  SHORTCUT COMMANDS  %%%%
\newcommand{\ds}{\displaystyle}
\newcommand{\Z}{\mathbb{Z}}
\newcommand{\arc}{\rightarrow}
\newcommand{\R}{\mathbb{R}}
\newcommand{\N}{\mathbb{N}}
\newcommand{\Q}{\mathbb{Q}}
\newcommand{\stirling}[2]{\genfrac{\{}{\}}{0pt}{}{#1}{#2}}

%%%%  footnote style %%%%

\renewcommand{\thefootnote}{\fnsymbol{footnote}}

\pagestyle{empty}
\begin{document}
	
	\noindent{\bf \large MATH 4410 }\\

        \noindent{\bf \large Nate Stott A02386053 }\\
	
	\noindent \underline{\hspace{2in}}\\
	
	{\bf Quiz \#5; Due 11:59 pm, 2/9/2024}\\

        \newpage
	\begin{enumerate}
		\item Please find closed formulae for the following recurrence relations.
		      \begin{enumerate}
			%\item $F_n  = F_{n-1} + F_{n-2}$ for $n>1$, and  $F_0  = 0$, $F_1=1$.
			\item $a_n = 4a_{n-1}-a_{n-2} - 6a_{n-3},$ for $n >2$, and $a_0  = 1$, $a_1  = 4$, and $a_2  = 15$.
                \end{enumerate} 
                Start \\
                \\
                The recurrence relation \\
                $$ a_0 = 1 $$
                $$ a_1 = 4 $$
                $$ a_2 = 15 $$
                Under the condition that $n > 2$ \\
                $$ a_n = 4 a_{n-1} - a_{n-2} -6a_{n-3} $$
                To make $a_n$ work for any value of $n$ I need to make a stipulation \\
                Define $a_n = 0$ if $n < 0$ \\
                Define $[n=0]$ will be 1 if true and 0 otherwise \\
                Here is the new recurrence relation, 
                $$ a_n = 4a_{n-1} - a_{n-2} - 6a_{n-3} + [n=0] $$
                Let's test it
                $$ a_0 = 4 a_{-1} - a_{-2} - 6 a_{-3} + [n=0] = 1 $$
                $$ a_1 = 4 a_{0} - a_{-1} - 6 a_{-2} + [n=0] = 4 $$
                $$ a_2 = 4 a_{1} - a_{0} - 6 a_{-1} + [n=0] = 15 $$
                Alright, so this works
                $$ a_n = 4a_{n-1} - a_{n-2} - 6a_{n-3} + [n=0] $$
                Define $ A(x) = \sum_{n\geq0} a_n x^n $
                $$ A(x) = \sum_{n\geq0} a_n x^n = 4 \sum_{n\geq0} a_{n-1} x^n - \sum_{n\geq0} a_{n-2} x^n - 6 \sum_{n\geq0} a_{n-3} x^n + \sum_{n\geq0} [n=0] x^n $$
                $$ = 4x \sum_{n\geq0} a_{n-1} x^{n-1} - x^2 \sum_{n\geq0} a_{n-2} x^{n-2} - 6x^3 \sum_{n\geq0} a_{n-3} x^{n-3} + 1 $$
                I checked that I can replace $\sum_{n\geq0} a_{n-c} x^{n-c}$ with $\sum_{n\geq0} a_n x^n$, for $c < 0 $
                $$ A(x) = 4x A(x) - x^2 A(x) - 6x^3 A(x) + 1 $$
                $$ A(x) - 4x A(x) + x^2 A(x) + 6x^3 A(x) = 1 $$
                $$ A(x)( 1 - 4x + x^2 + 6x^3 ) = 1 $$
                $$ A(x) = \frac{1}{( 1 - 4x + x^2 + 6x^3 )} $$
                Partial fraction decomposition
                $$ A(x) = \frac{1}{( 1 - 4x + x^2 + 6x^3 )} = \frac{1}{(x+1)(2x-1)(3x-1)} = \frac{B}{(x+1)} + \frac{C}{(2x-1)} + \frac{D}{(3x-1)} $$
                I did the math on a white board $B=\frac{1}{12}$, $C=\frac{4}{3}$, and $D=\frac{-9}{4}$
                $$ A(x) = \frac{1}{12} \frac{1}{(x+1)} + \frac{4}{3} \frac{1}{(2x-1)} - \frac{9}{4} \frac{1}{(3x-1)} $$
                $$ = \frac{1}{12} (\frac{1}{1-(-x)}) - \frac{4}{3} (\frac{1}{1-(2x)}) + \frac{9}{4} (\frac{1}{1-(3x)}) $$
                $$ = \frac{1}{12} ( \sum_{n\geq0} (-1)^n x^n ) - \frac{4}{3} ( \sum_{n\geq0} (2)^n x^n ) + \frac{9}{4} ( \sum_{n\geq0} (3)^n x^n ) $$
                Dropping the sums, and $x^n$ you get
                $$ a_n = \frac{1}{12} ( (-1)^n ) - \frac{4}{3} ( (2)^n ) + \frac{9}{4} ( (3)^n ) $$
                $$ = \frac{(-1)^n}{12} - \frac{4(2)^n}{3} + \frac{9(3)^n}{4} $$
                $$ = \frac{(-1)^n}{12} - \frac{16(2)^n}{12} + \frac{27(3)^n}{12} $$
                $$ = \frac{(-1)^n - 16(2)^n + 27(3)^n}{12} $$
                There you have it
                $$ a_n = \frac{(-1)^n - 16(2)^n + 27(3)^n}{12} $$
                Checked with n = 0, 1, 2, 3 \\
                
                End
                
                \newpage
                \begin{enumerate}[resume]
			\item $L_n = L_{n-1} + L_{n-2}$, for $n > 1$, and $L_0 = 2,$ and $L_1= 1$. (This recurrence defines the 
			\emph{Lucas numbers,} which are very closely related to the Fibonacci numbers.)
		      \end{enumerate} 

                Start \\

                The recurrence relation
                $$ L_0 = 2 $$
                $$ L_1 = 1 $$
                Under the condition that $n>1$
                $$ L_n = L_{n-1} + L_{n-2} $$
                To allow n to be any integer \\
                Define $[n=0]$ will give $2$ if true and $0$ otherwise \\
                Define $[n=1]$ will give $-1$ if true and $0$ otherwise \\
                $$ L_n = L_{n-1} + L_{n-2} + [n=0] + [n=1] $$
                Test
                $$ L_0 = L_{-1} + L_{-2} + 2 + 0 = 2 $$
                $$ L_1 = L_{0} + L_{-1} + 0 + -1 = 1 $$
                So it works
                $$ L_n = L_{n-1} + L_{n-2} + [n=0] + [n=1] $$
                Define $L(x) = \sum_{n\geq0} L_n x^n $ \\
                Let's get the summation and $x^n$ in there and get the generating function
                $$ L(x) = \sum_{n\geq0} L_n x^n = \sum_{n\geq0} L_{n-1} x^n + \sum_{n\geq0} L_{n-2} x^n + \sum_{n\geq0} [n=0] x^n + \sum_{n\geq0} [n=1] x^n $$
                $$ = \sum_{n\geq0} L_{n-1} x^n + \sum_{n\geq0} L_{n-2} x^n + 2 - x $$
                $$ = x \sum_{n\geq0} L_{n-1} x^{n-1} + x^2 \sum_{n\geq0} L_{n-2} x^{n-2} + 2 - x $$
                I checked $\sum_{n\geq0} L_{n-c} x^{n-c} = \sum_{n\geq0} L_n x^n $ for $c<0$ \\
                $$ L(x) = x L(x) + x^2 L(x) + 2 - x $$
                $$ L(x) - x L(x) - x^2 L(x) = 2 - x $$
                $$ L(x) ( 1 - x - x^2 ) = 2 - x $$
                $$ L(x) = \frac{2 - x}{( 1 - x - x^2 )} $$
                Alright so here is the generating function,
                $$ L(x) = \frac{2-x}{( 1 - x - x^2 )} $$
                Find the reverse polynomial of $ 1 - x - x^2 $
                $$ p(x) = 1 - x - x^2 $$
                $$ p^R(x) = x^2 - x - 1 $$
                The roots of $p^R(x)$ are $\frac{1-\sqrt{5}}{2}$ and $\frac{1+\sqrt{5}}{2}$ \\
                Thus \\
                $$ 1 - x - x^2 = (1-(\frac{1-\sqrt{5}}{2})x)(1-(\frac{1+\sqrt{5}}{2})x) $$
                To clean things up, let $p = \frac{1-\sqrt{5}}{2}$ and $q = \frac{1+\sqrt{5}}{2}$
                $$ 1 - x - x^2 = (1-px)(1-qx) $$
                So
                $$ L(x) = \frac{2-x}{( 1 - x - x^2 )}= \frac{2-x}{(1-px)(1-qx)}$$
                Partial fraction time,
                $$ \frac{2-x}{(1-px)(1-qx)} = \frac{A}{(1-px)} + \frac{B}{(1-qx)} $$
                $$ 2 - x = A(1-qx) + B(1-px) $$
                $$ 2 - x = A-Aqx + B-Bpx $$
                Implying
                $$ 2 = A + B $$
                $$ 1 = Aq + Bp $$
                Solve for A
                $$ 2 - B = A $$
                Plug in 
                $$ 1 = (2 - B)q + Bp $$
                $$ 1 = 2q - Bq + Bp $$
                $$ 1 - 2q = - Bq + Bp $$
                $$ 1 - 2q = B (-q + p) $$
                $$ \frac{1 - 2q}{(-q + p)} = B  $$
                $$ 2 - \frac{1 - 2q}{(-q + p)} = A $$
                Check for $ 2 = A + B $
                $$ 2 = (2 - \frac{1 - 2q}{(-q + p)}) + (\frac{1 - 2q}{(-q + p)})$$
                $$ 2 (-q + p) = 2(-q + p) - 1 - 2q + 1 - 2q$$
                $$ 2 (-q + p) = 2(-q + p)$$
                $$ 2 = 2$$
                It worked for the first equation! \\
                Check for $ 1 = Aq + Bp $
                $$ 1 = (2 - \frac{1 - 2q}{(-q + p)})q + (\frac{1 - 2q}{(-q + p)})p $$
                $$ 1 = 2q - \frac{q(1 - 2q)}{(-q + p)} + \frac{p(1 - 2q)}{(-q + p)} $$
                $$ (-q + p) = 2q(-q + p) - q(1 - 2q) + p(1 - 2q) $$
                $$ -q + p = -2q^2 + 2qp - q + 2q^2 + p - 2qp $$
                $$ -q + p = - q + p $$
                It worked for the second equation! \\
                So
                $$ A = 2 - \frac{1 - 2q}{(-q + p)} $$
                $$ B = \frac{1 - 2q}{(-q + p)}  $$
                Plug in
                $$ $$
                $$ L(x) = \frac{2-x}{( 1 - x - x^2 )}= \frac{2-x}{(1-px)(1-qx)} = \frac{A}{(1-px)} + \frac{B}{(1-qx)} = \frac{(2 - \frac{1 - 2q}{(-q + p)})}{(1-px)} + \frac{(\frac{1 - 2q}{(-q + p)})}{(1-qx)}$$
                $$ = (2 - \frac{1 - 2q}{(-q + p)})\frac{1}{(1-px)} + (\frac{1 - 2q}{(-q + p)})\frac{1}{(1-qx)} $$
                Changing the generating function to summations
                $$ = (2 - \frac{1 - 2q}{(-q + p)}) \sum_{n\geq0} (p)^n x^n + (\frac{1 - 2q}{(-q + p)})  \sum_{n\geq0} (q)^n x^n $$
                The indexes line up so now I can throw the summations and $x^n$ off and get $L_n$
                $$ L_n = (2 - \frac{1 - 2q}{(-q + p)}) (p)^n + (\frac{1 - 2q}{(-q + p)}) (q)^n $$
                $$ L_n = (2 - \frac{1 - 2q}{(p-q)}) (p)^n + (\frac{1 - 2q}{(p-q)}) (q)^n $$
                $$ L_n = 2p^n - (p^n)\frac{1 - 2q}{p-q}  + (q^n)\frac{1 - 2q}{p-q} $$ 
                $$ L_n = 2p^n + (q^n)\frac{1 - 2q}{p-q} - (p^n)\frac{1 - 2q}{p-q} $$ 
                $$ L_n = 2p^n + \frac{1 - 2q}{p-q} (q^n - p^n) $$ 

                With $p = \frac{1-\sqrt{5}}{2}$ and $q = \frac{1+\sqrt{5}}{2}$ the closed formula for $L_n$ is,
                $$ L_n = 2p^n + \frac{1 - 2q}{p-q} (q^n - p^n) $$ 
                
                Checked for n = 0, 1, 2, 3
                
                End
        
	\end{enumerate}

        \newpage
        \begin{enumerate}[resume]
		\item Use generating functions to prove the identities:
		\begin{enumerate}
			\item $5F_n + L_n = 2L_{n+1}$, for $L_n$ as defined in \#1 (b), and $F_n$ is the n-th Fibonacci number.
            \end{enumerate} 

            Start
            
            $$ L_0 = 2 $$
            $$ L_1 = 1 $$
            Under the condition that $n>1$
            $$ L_n = L_{n-1} + L_{n-2} $$

            $$ F_0 = 0 $$
            $$ F_1 = 1 $$
            Under the condition that $n>1$
            $$ F_n = F_{n-1} + F_{n-2} $$

            The generating function for these recurrences are
            $$ L(x) = \frac{2-x}{ 1 - x - x^2 } $$
            $$ F(x) = \frac{x}{1-x-x^2} $$

            Prove this relation
            $$ 5F_n + L_n = 2L_{n+1} $$
            Manipulate a little
            $$ L_n = 2L_{n+1} - 5F_n $$
            $$ L_n - 2L_{n+1} = - 5F_n $$
            $$ L_n - 2 L_{n} - 2 L_{n-1} = - 5F_n $$
            $$ L_n(1 - 2) = - 5F_n  + 2 L_{n-1} $$
            $$ -L_n = - 5F_n  + 2 L_{n-1} $$
            $$ L_n = 5F_n  - 2 L_{n-1} $$
            If I can get the generating function for $ L_n = 5F_n  - 2 L_{n-1} $ and show that it's the same generating function as, $\frac{2-x}{ 1 - x - x^2 }$ then I will have proved the relation \\
            \\
            find the generating function to 
            $$ L_n = 5F_n  - 2 L_{n-1} $$
            make the recurrence work for any n. \\
            Define $n<0$ to be 0 \\
            Define $[n=0]$ will give 2 if true, otherwise 0

            $$ L_n = 5F_n  - 2 L_{n-1} + [n=0] $$
            Test it
            $$ L_0 = 5 F_0 - 2 L_{-1} + [n=0] = 5(0) - 2 (0) + 2 = 2 $$
            $$ L_1 = 5 F_1 - 2 L_{0} + [n=0] = 5(1) - 2 (2) + 0 = 5 - 4 = 1 $$
            It works \\
            Define $ G(x) = \sum_{n\geq0} L_n x^n $ \\
            Multiply by the sum and $x^n$
            $$ G(x) = \sum_{n\geq0} L_n x^n = 5 \sum_{n\geq0} F_n x^n - 2 \sum_{n\geq0} L_{n-1} x^n + \sum_{n\geq0} [n=0] x^n $$
            $$ = 5 \sum_{n\geq0} F_n x^n - 2x \sum_{n\geq0} L_{n-1} x^{n-1} + 2 $$
            $$ G(x) = 5 F(x) - 2x G(x) + 2 $$
            $$ G(x) + 2x G(x) = 5 F(x) + 2 $$
            $$ G(x) (1 + 2x) = 5 F(x) + 2 $$
            $$ G(x) = \frac{5 F(x) + 2}{(1 + 2x)} $$
            Alright so I need to show that $G(x) = L(x)$ with $ L(x) = \frac{2-x}{ 1 - x - x^2 } $
            $$ G(x) = \frac{5 (\frac{x}{1-x-x^2}) + 2}{(1 + 2x)} $$
            $$ G(x) = \frac{ \frac{5x}{1-x-x^2} + \frac{2-2x-2x^2}{1-x-x^2}}{(1 + 2x)} $$
            $$ G(x) = \frac{ \frac{ 2+ 3x - 2x^2}{1-x-x^2}}{(1 + 2x)} $$
            $$ G(x) = \frac{ 2+ 3x - 2x^2}{1-x-x^2} \frac{1}{(1 + 2x)} $$
            $$ G(x) = \frac{ 2+ 3x - 2x^2}{(1-x-x^2)(1 + 2x)} $$
            $$ G(x) = \frac{ -(2x+1)(x-2)}{(1-x-x^2)(2x + 1 )} $$
            $$ G(x) = \frac{ -(x-2)}{(1-x-x^2)} $$
            $$ G(x) = \frac{ 2-x}{1-x-x^2} $$
            Well 
            $$ L(x) = \frac{2-x}{ 1 - x - x^2 } $$
            $$ G(x) = \frac{ 2-x}{1-x-x^2} $$
            So they are equal
            $$ G(x) = \frac{ 2-x}{1-x-x^2} = L(x) $$
            This means that $5F_n + L_n = 2L_{n+1} $ is a true relation. Because they have the same generating function. Which means that the coefficients are the same. Which means they have the same closed formulae.
            
            End

            \newpage
            \begin{enumerate}[resume]
			\item $F_n +L_n =2F_{n+1}$, $L_n$ and $F_n$ as above.
	    \end{enumerate} 

            Start
            
            $$ L_0 = 2 $$
            $$ L_1 = 1 $$
            Under the condition that $n>1$
            $$ L_n = L_{n-1} + L_{n-2} $$

            $$ F_0 = 0 $$
            $$ F_1 = 1 $$
            Under the condition that $n>1$
            $$ F_n = F_{n-1} + F_{n-2} $$

            The generating function for these recurrences are
            $$ L(x) = \frac{2-x}{ 1 - x - x^2 } $$
            $$ F(x) = \frac{x}{1-x-x^2} $$

            I need to prove this relation
            $$ F_n +L_n =2F_{n+1} $$

            Manipulate a little
            $$ F_n +L_n =2F_{n+1} $$
            $$ F_n - 2F_{n+1} = - L_n $$
            $$ F_n - 2(F_{n} + F_{n-1}) = - L_n $$
            $$ F_n - 2F_{n} - 2 F_{n-1} = - L_n $$
            $$ F_n - 2F_{n} = 2 F_{n-1} - L_n $$
            $$ F_n (1 - 2) = 2 F_{n-1} - L_n $$
            $$ - F_n = 2 F_{n-1} - L_n $$
            $$ F_n = L_n - 2 F_{n-1} $$

            If I can get the generating function for $ F_n = L_n - 2 F_{n-1} $ and show that it's the same generating function as, $\frac{x}{ 1 - x - x^2 }$ then I will have proved the relation \\
            \\
            find the generating function to 
            $$ F_n = L_n - 2 F_{n-1} $$
            make the recurrence work for any n. \\
            Define $n<0$ to be 0 \\
            Define $[n=0]$ will give -2 if true, otherwise 0
            $$ F_n = L_n - 2 F_{n-1} + [n=0] $$
            Test it 
            $$ F_0 = L_0 - 2 F_{-1} + [n=0] = 2 - 2(0) - 2 = 0 $$
            $$ F_1 = L_1 - 2 F_{0} + [n=0] = 1 + 2(0) + 0 = 1 $$
            It works \\
            Define $ H(x) = \sum_{n\geq0} F_n x^n $ \\
            Multiply by the sum and $x^n$
            $$ H(x) = \sum_{n\geq0} F_n x^n = \sum_{n\geq0} L_n x^n - 2 \sum_{n\geq0} F_{n-1} x^n + \sum_{n\geq0} [n=0] x^n $$
            $$ = \sum_{n\geq0} L_n x^n - 2x \sum_{n\geq0} F_{n-1} x^{n-1} - 2 $$
            $$ H(x) = L(x) - 2x H(x) - 2 $$
            $$ H(x) + 2x H(x) = L(x) - 2 $$
            $$ H(x)(1 + 2x) = L(x) - 2 $$
            $$ H(x) = \frac{L(x) - 2}{1 + 2x} $$
            $$ H(x) = \frac{\frac{2-x}{ 1 - x - x^2 } - 2}{1 + 2x} $$
            $$ H(x) = \frac{\frac{2-x}{ 1 - x - x^2 } - \frac{2(1 - x - x^2)}{1 - x - x^2} }{1 + 2x} $$
            $$ H(x) = \frac{\frac{2-x}{ 1 - x - x^2 } + \frac{-2 + 2x + 2x^2}{1 - x - x^2} }{1 + 2x} $$
            $$ H(x) = \frac{\frac{2-x-2 + 2x + 2x^2}{ 1 - x - x^2 } }{1 + 2x} $$
            $$ H(x) = \frac{2-x-2 + 2x + 2x^2}{ (1 + 2x)(1 - x - x^2) }  $$
            $$ H(x) = \frac{x + 2x^2}{ (1 + 2x)(1 - x - x^2) }  $$
            $$ H(x) = \frac{x(1+ 2x)}{ (1 + 2x)(1 - x - x^2) }  $$
            $$ H(x) = \frac{x}{ (1 - x - x^2) } $$
            Well 
            $$ H(x) = \frac{x}{ (1 - x - x^2) } $$
            $$ F(x) = \frac{x}{1-x-x^2} $$
            So they are equal,
            $$ H(x) = \frac{x}{1-x-x^2} = F(x) $$
            This means that $ F_n +L_n =2F_{n+1} $ is a true relation. Because they have the same generating function. Which means that the coefficients are the same. Which means they have the same closed formulae.
            
            End


     
        \end{enumerate}
		
        
        \newpage
        \begin{enumerate}[resume]
		\item Use generating functions and a computer to determine the number of ways to make \$1.00 using pennies, nickels, dimes, quarters, and a new denomination called the \emph{Euclid} which has value \$0.28.
	\end{enumerate}
        
        $p$ Pennies 0.01 \\
        $n$ Nickels 0.05 \\
        $d$ Dimes 0.10 \\
        $q$ Quarters 0.25 \\
        $u$ Euclid 0.28 \\
        $$ 1p + 5n + 10d + 25q + 28u = 100 $$
        Find a generating function such that $ p(x) n(x) d(x) q(x) u(x) $ expanded out to the 100th term is the number of ways to make \$1.00
	$$ p(x) = \frac{1}{1-x} = x^0+x^1+x^2+x^3+... $$
        $$ n(x) = \frac{1}{(1-x)^5} = x^0+x^5+x^{10}+x^{15}+... $$
        $$ d(x) = \frac{1}{(1-x)^{10}} = x^0+x^{10}+x^{20}+x^{30}+... $$
	$$ q(x) = \frac{1}{(1-x)^{25}} = x^0+x^{25}+x^{50}+x^{75}+... $$
	$$ u(x) = \frac{1}{(1-x)^{28}} = x^0+x^{28}+x^{56}+x^{84}+... $$
        Getting the generating function I am after
        $$ (\frac{1}{1-x}) (\frac{1}{(1-x)^5}) (\frac{1}{(1-x)^{10}}) (\frac{1}{(1-x)^{25}}) (\frac{1}{(1-x)^{28}}) $$
        Through that into the computer
        $$ taylor(1/((1-x)*(1-x^5)*(1-x^{10})*(1-x^{25})*(1-x^{28})),x,0,101); $$
        And you get
        $$ ... + 353*x^{99}+382*x^{100}+390*x^{101} + ... $$
        Thus there are $382$ ways to make \$1.00 out of \$0.01, \$0.05, \$0.10, \$0.25, and \$0.28 denominations. 



 
	\noindent \underline{\hspace{3in}}\\
	
	
	
	
\end{document}