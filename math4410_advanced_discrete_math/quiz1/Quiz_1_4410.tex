\documentclass[10pt, AMS Euler]{article}
\textheight=9.25in \textwidth=7in \topmargin=-.75in
\oddsidemargin=-0.25in
\evensidemargin=-0.25in
\usepackage{url}  % The bib file uses this
\usepackage{graphicx} %to import pictures
\usepackage{amsmath, amssymb, color, wasysym}
\usepackage{theorem, concrete, multicol,tikz}
\usepackage[normalem]{ulem} %for strikethrough (\sout{blah})
\usepackage{enumitem}


\setlength{\intextsep}{5mm} \setlength{\textfloatsep}{5mm}
\setlength{\floatsep}{5mm}


{\theorembodyfont{\rmfamily}
	\newtheorem{definition}{Definition}[section]}
{\theorembodyfont{\rmfamily} \newtheorem{example}{Example}[section]}
{\theorembodyfont{\rmfamily} \newtheorem{lemma}{Lemma}[section]}
{\theorembodyfont{\rmfamily} \newtheorem{theorem}{Theorem}[section]}
{\theorembodyfont{\rmfamily} \newenvironment{proof}{\par{\it
			Proof:}}{\nopagebreak[4]\rule{2mm}{2mm}}}
{\theorembodyfont{\rmfamily}
	\newenvironment{solution}{\par{\bf{Solution:}}}{\nopagebreak[4]\rule{2mm}{2mm}}}

\usetikzlibrary{arrows}
\usetikzlibrary{shapes}
\newcommand{\mymk}[1]{%
	\tikz[baseline=(char.base)]\node[anchor=south west, draw,rectangle, rounded corners, inner sep=2pt, minimum size=7mm,
	text height=2mm](char){\ensuremath{#1}} ;}

\newcommand*\circled[1]{\tikz[baseline=(char.base)]{
		\node[shape=circle,draw,inner sep=2pt] (char) {#1};}}


%%%%  SHORTCUT COMMANDS  %%%%
\newcommand{\ds}{\displaystyle}
\newcommand{\Z}{\mathbb{Z}}
\newcommand{\arc}{\rightarrow}
\newcommand{\R}{\mathbb{R}}
\newcommand{\N}{\mathbb{N}}
\newcommand{\Q}{\mathbb{Q}}
\newcommand{\stirling}[2]{\genfrac{\{}{\}}{0pt}{}{#1}{#2}}

%%%%  footnote style %%%%

\renewcommand{\thefootnote}{\fnsymbol{footnote}}

\pagestyle{empty}
\begin{document}

	\noindent{\bf \large MATH 4410 }\\
    \noindent{\bf \large Nate Stott A02386053 }\\
	\noindent \underline{\hspace{2in}}\\
 
	{\bf Quiz \#1; Due 11:59 pm, 1/16/2024}\\
 
        \newpage
    	\begin{enumerate}
        \item Write $n^4$ as a sum of falling powers: $n^4 = c_0\cdot n^{\underline{0}} + c_1 n^{\underline{1}}+c_2 n^{\underline{2}} +c_3 n^{\underline{3}} +c_4 n^{\underline{4}}$.
        \end{enumerate}
            
            I used a python program I made to make the difference table. 
            Here is the code on github if you want to see: 
            \newline \newline
            \url{https://github.com/funkybooboo/MATH4110_DifferanceTable} 
            \newline \newline
            The program produced the numbers in the following table. However I don't know Latex that well so I copied the table you made and checked that our numbers matched then edited it to look how I like.
            \begin{center}
                \begin{tabular}{l|ccccccccc}
                    {\bf $n$ } & 0 & 1 & 2 & 3 & 4 & 5 & 6 & 7  \\
                    \hline
                    {$\Delta^{(0)} n^4$} & 0 & 1 & 16 & 81 & 256 & 625 & 1296 & 2401   \\
                    {$\Delta^{(1)} n^4$} & 1& 15 & 65 & 175 & 369 & 671 & 1105 &    \\
                    {$\Delta^{(2)}n^4$} &  14 & 50 & 110 & 194 & 302 & 434 &  &   \\
                    {$\Delta^{(3)}n^4$} & 36 & 60 & 84 & 108 & 132 &  &  &    \\
                    {$\Delta^{(4)}n^4$} & 24 & 24 & 24 & 24 &  &  &  &    \\
                    {$\Delta^{(5)}n^4$} & 0 & 0 & 0 &  &  &  &  &    \\
                \end{tabular}
            \end{center}
            Then I wanted to prove that this statement is true to make sure you weren't pulling any trickery.
            $$\ds\binom{n}{k} = \frac{n!}{k!(n-k)!} = \frac{n^{\underline{k}}}{k!}$$
            Start proof.
            \newline
            Lets focus in.
            $$ \frac{n!}{k!(n-k)!} = \frac{n^{\underline{k}}}{k!} $$
            We can eliminate $k!$.
            $$ \frac{n!}{(n-k)!} = n^{\underline{k}} $$
            Wait $\frac{n!}{(n-k)!} = n^{\underline{k}}$ is the definition of $n^{\underline{k}}$ so I guess you weren't pulling anything there.
            \newline
            End proof.
            \newline
            Lets use The Discrete Taylor Theorem to find a falling power representation for $n^4$.
            $$f(n)=\sum_{k \geq 0}\Delta^{(k)}f(a)\ds\binom{n-a}{k}$$
            Let $a=0$ so that we can get the first item of every row on the difference table.
            $$f(n)=n^4=\sum_{k \geq 0}\Delta^{(k)}f(0)\ds\binom{n}{k}=0\ds\binom{n}{0}+1\ds\binom{n}{1}+14\ds\binom{n}{2}+36\ds\binom{n}{3}+24\ds\binom{n}{4}+0\ds\binom{n}{5}+0\ds\binom{n}{6}+0\ds\binom{n}{7}+...$$
            We can through out the terms with 0 as the leading coefficient.
            $$=1\ds\binom{n}{1}+14\ds\binom{n}{2}+36\ds\binom{n}{3}+24\ds\binom{n}{4}$$
            OK so now I can use $\ds\binom{n}{k} = \frac{n^{\underline{k}}}{k!}$ to my advantage. 
            $$=1(\frac{n^{\underline{1}}}{1!})+14(\frac{n^{\underline{2}}}{2!})+36(\frac{n^{\underline{3}}}{3!})+24(\frac{n^{\underline{4}}}{4!})$$
            Simplifying.
            $$=n^{\underline{1}}+14(\frac{n^{\underline{2}}}{2})+36(\frac{n^{\underline{3}}}{6})+24(\frac{n^{\underline{4}}}{24})$$
            $$=n^{\underline{1}}+7n^{\underline{2}}+6n^{\underline{3}}+n^{\underline{4}}$$
            There you have it.
            $$n^4=n^{\underline{1}}+7n^{\underline{2}}+6n^{\underline{3}}+n^{\underline{4}}$$
            
        \newpage
        \begin{enumerate}[resume]
		\item Create a function that is a polynomial in $n$ (ideally with no falling powers, but simplifying is not necessary) for the $\ds\sum_{i=0}^n i^4$. 
        \end{enumerate}

            $$\ds\sum_{i=0}^n i^4=\ds\sum_{i=0}^n (i^{\underline{1}}+7i^{\underline{2}}+6i^{\underline{3}}+i^{\underline{4}})$$
            I know the summation doesn't distribute but you know what I'm doing
            $$=\ds\sum_{i=0}^n i^{\underline{1}}+\ds\sum_{i=0}^n 7i^{\underline{2}}+\ds\sum_{i=0}^n 6i^{\underline{3}}+\ds\sum_{i=0}^n i^{\underline{4}}$$
            $$=\ds\left .\frac{i^{\underline{2}}}{2}\right|_{i=0}^{i=n+1} + 7(\ds\left .\frac{i^{\underline{3}}}{3}\right|_{i=0}^{i=n+1}) + 6(\ds\left .\frac{i^{\underline{4}}}{4}\right|_{i=0}^{i=n+1}) + \ds\left .\frac{i^{\underline{5}}}{5}\right|_{i=0}^{i=n+1} $$
            $$=(\frac{(n+1)^{\underline{2}}}{2}-\frac{(0)^{\underline{3}}}{3})+7(\frac{(n+1)^{\underline{3}}}{3}-\frac{(0)^{\underline{3}}}{3})+6(\frac{(n+1)^{\underline{4}}}{4}-\frac{(0)^{\underline{4}}}{4})+(\frac{(n+1)^{\underline{5}}}{5}-\frac{(0)^{\underline{5}}}{5})$$
            $$=\frac{(n+1)^{\underline{2}}}{2}+7(\frac{(n+1)^{\underline{3}}}{3})+6(\frac{(n+1)^{\underline{4}}}{4})+\frac{(n+1)^{\underline{5}}}{5}$$
            $$=\frac{(n+1)(n)}{2}+7(\frac{(n+1)(n)(n-1)}{3})+6(\frac{(n+1)(n)(n-1)(n-2)}{4})+\frac{(n+1)(n)(n-1)(n-2)(n-3)}{5}$$
            $$=\frac{n^2+n}{2}+7(\frac{n^3-n}{3})+6(\frac{n^4-2n^3-n+2}{4})+\frac{n^5-5n^4+5n^3+5n^2-6n}{5}$$
            Simplify
            $$=\frac{n^5}{5}+\frac{n^4}{2}+\frac{n^3}{3}+\frac{3n^2}{2}-\frac{68n}{15}+3$$
            There you have it
            $$\ds\sum_{i=0}^n i^4 = \frac{n^5}{5}+\frac{n^4}{2}+\frac{n^3}{3}+\frac{3n^2}{2}-\frac{68n}{15}+3$$

        \newpage
		\begin{enumerate}[resume]
		\item Show that $\ds\sum_{i = 0}^n (4i^3 +6i^2+4i+1) = (n+1)^4$.
        \end{enumerate}
            Start proof
            
            $$\ds\sum_{i = 0}^n (4i^3 +6i^2+4i+1) = (n+1)^4 $$
            $$\ds\sum_{i = 0}^n (4i^3 +6i^2+4i+1) = n^4+4n^3+6n^2+4n+1 $$
            focus in on left side
            $$=\ds\sum_{i = 0}^n (4(i^{\underline{3}}+3i^{\underline{2}}+i^{\underline{1}})+6(i^{\underline{2}}+i^{\underline{1}})+4(i^{\underline{1}})+1) $$
            I know the summation doesn't distribute but you know
            $$=\ds\sum_{i = 0}^n 4(i^{\underline{3}}+3i^{\underline{2}}+i^{\underline{1}})+\ds\sum_{i = 0}^n 6(i^{\underline{2}}+i^{\underline{1}})+\ds\sum_{i = 0}^n 4(i^{\underline{1}})+\ds\sum_{i = 0}^n 1$$
            $$=4 \ds\sum_{i = 0}^n (i^{\underline{3}}+3i^{\underline{2}}+i^{\underline{1}})+6 \ds\sum_{i = 0}^n (i^{\underline{2}}+i^{\underline{1}})+4 \ds\sum_{i = 0}^n i^{\underline{1}}+n+1$$
            $$=4( \ds\sum_{i = 0}^n i^{\underline{3}}+ 3\ds\sum_{i = 0}^n i^{\underline{2}}+\ds\sum_{i = 0}^n i^{\underline{1}})+6 (\ds\sum_{i = 0}^n i^{\underline{2}}+\ds\sum_{i = 0}^n i^{\underline{1}})+4 \ds\sum_{i = 0}^n i^{\underline{1}}+n+1$$
            $$=4( \ds\left .\frac{i^{\underline{4}}}{4}\right|_{i=0}^{i=n+1} + 3 (\ds\left .\frac{i^{\underline{3}}}{3}\right|_{i=0}^{i=n+1})+\ds\left .\frac{i^{\underline{2}}}{2}\right|_{i=0}^{i=n+1})+ 6 (\ds\left .\frac{i^{\underline{3}}}{3}\right|_{i=0}^{i=n+1}+\ds\left .\frac{i^{\underline{2}}}{2}\right|_{i=0}^{i=n+1})+4 (\ds\left .\frac{i^{\underline{2}}}{2}\right|_{i=0}^{i=n+1})+n+1$$
            $$=4( \frac{(n+1)^{\underline{4}}}{4} + 3 (\frac{(n+1)^{\underline{3}}}{3})+\frac{(n+1)^{\underline{2}}}{2})+ 6 (\frac{(n+1)^{\underline{3}}}{3}+\frac{(n+1)^{\underline{2}}}{2}+4 (\frac{(n+1)^{\underline{2}}}{2})+n+1$$
            $$=4(\frac{(n+1)n(n-1)(n-2)}{4}+3(\frac{(n+1)n(n-1)}{3})+\frac{(n+1)(n)}{2})+6(\frac{(n+1)n(n-1)}{3}+\frac{(n+1)n}{2})+4(\frac{(n+1)n}{2})+n+1$$
            $$=n^4+4n^3+6n^2+4n+1$$
            There you go
            $$\ds\sum_{i = 0}^n (4i^3 +6i^2+4i+1) = (n+1)^4 $$
            End proof
            
            
            
        
        \newpage
        \begin{enumerate}[resume]
		\item Compute $\stirling{7}{3}$ using the difference table for $n^7$.
        \end{enumerate}
        
		      \begin{center}
                \begin{tabular}{l|ccccccccc}
                    {\bf $n$ } & 0 & 1 & 2 & 3 & 4 & 5 & 6 & 7 & 8\\
                    \hline
                    {$\Delta^{(0)} n^7$} & 0 & 1 & 128 & 2187 & 16384 & 78125 & 279936 & 823543 & 2097152 \\
                    {$\Delta^{(1)} n^7$} & 1 & 127 & 2059 & 14197 & 61741 & 201811 & 543607 &  1273609  \\
                    {$\Delta^{(2)}n^7$} & 126 & 1932 & 12138 & 47544 & 140070 & 341796 &  730002 \\
                    {$\Delta^{(3)}n^7$} & 1806 & 10206 & 35406 & 92526 & 201726 &  388206  \\
                    {$\Delta^{(4)}n^7$} & 8400 & 25200 & 57120 & 109200 &   186480  \\
                    {$\Delta^{(5)}n^7$} & 16800 & 31920 & 52080 &  77280   \\
                    {$\Delta^{(6)}n^7$} & 15120 & 20160 & 25200  \\
                    {$\Delta^{(7)}n^7$} & 5040 & 5040   \\
                    {$\Delta^{(8)}n^7$} & 0 &   \\
                \end{tabular}
            \end{center}
            Given
            $$\stirling{k}{j} = \frac{\Delta^{(j)}[(0)^k]}{j!}$$
            Compute
            $$\stirling{7}{3} = \frac{\Delta^{(3)}[(0)^7]}{3!}=\frac{1806}{3!}=301$$
            
	
	
	\noindent \underline{\hspace{3in}}\\
	
	
	
	
\end{document}