\documentclass[10pt, AMS Euler]{article}
\textheight=9.25in \textwidth=7in \topmargin=-.75in
\oddsidemargin=-0.25in
\evensidemargin=-0.25in
\usepackage{url}  % The bib file uses this
\usepackage{graphicx} %to import pictures
\usepackage{amsmath, amssymb, color, wasysym}
\usepackage{theorem, concrete, multicol,tikz}
\usepackage[normalem]{ulem} %for strikethrough (\sout{blah})
\usepackage{enumitem}
\usepackage{amsmath}

\setlength{\intextsep}{5mm} \setlength{\textfloatsep}{5mm}
\setlength{\floatsep}{5mm}


{\theorembodyfont{\rmfamily}
	\newtheorem{definition}{Definition}[section]}
{\theorembodyfont{\rmfamily} \newtheorem{example}{Example}[section]}
{\theorembodyfont{\rmfamily} \newtheorem{lemma}{Lemma}[section]}
{\theorembodyfont{\rmfamily} \newtheorem{theorem}{Theorem}[section]}
{\theorembodyfont{\rmfamily} \newenvironment{proof}{\par{\it
			Proof:}}{\nopagebreak[4]\rule{2mm}{2mm}}}
{\theorembodyfont{\rmfamily}
	\newenvironment{solution}{\par{\bf{Solution:}}}{\nopagebreak[4]\rule{2mm}{2mm}}}

\usetikzlibrary{arrows}
\usetikzlibrary{shapes}
\newcommand{\mymk}[1]{%
	\tikz[baseline=(char.base)]\node[anchor=south west, draw,rectangle, rounded corners, inner sep=2pt, minimum size=7mm,
	text height=2mm](char){\ensuremath{#1}} ;}

\newcommand*\circled[1]{\tikz[baseline=(char.base)]{
		\node[shape=circle,draw,inner sep=2pt] (char) {#1};}}


%%%%  SHORTCUT COMMANDS  %%%%
\newcommand{\ds}{\displaystyle}
\newcommand{\Z}{\mathbb{Z}}
\newcommand{\arc}{\rightarrow}
\newcommand{\R}{\mathbb{R}}
\newcommand{\N}{\mathbb{N}}
\newcommand{\Q}{\mathbb{Q}}
\newcommand{\stirling}[2]{\genfrac{\{}{\}}{0pt}{}{#1}{#2}}

%%%%  footnote style %%%%

\renewcommand{\thefootnote}{\fnsymbol{footnote}}

\pagestyle{empty}
\begin{document}
	
	\noindent{\bf \large MATH 4410 }\\
    Nate Stott A02386053
	
	\noindent \underline{\hspace{2in}}\\
	
	{\bf Quiz \#2; Due 11:59 pm, 1/19/2024}\\

        \newpage
    	\begin{enumerate}
		\item Let $S_n$ denote the maximum number of regions into which  3-dimensional space is partitioned by planes so that no three of the cutting planes are colinear and no four are concurrent (no four intersect in a point).  Find a formula for $S_n$ in terms of binomial coefficients or powers of $n$.
		\emph{Hint/Suggestion:} Show that $\Delta S_n = P_n$, where $P_n$ is the maximum number of regions into which 2-dimensional space is partitioned by $n$ lines, no three concurrent.
        \end{enumerate}

            An argument for $ \Delta S_n = P_n  $ where $P_n$ is the 2d version of this 3d problem: The statement means that 2d space can be thought to be the derivative of 3d space. Why? If you have a 2d function then the slope of the tangent line at any given point is the derivative. Thus for 2d space a 1d line is needed to describe the derivative. If you have a 3d function then the slope of the tangent plane at any given point is the derivative. Thus for 3d space a 2d plane is needed to describe the derivative. Thus 2d space is the derivative of 3d space. For this particular problem it also intuitively makes sense. $P_n$ is the maximum number of regions into which 2d space is partitioned by lines so that no two lines are linear and no three intersect in a point. Taking the integral of $P_n$ means to sum up layers to get 3d space. Thus you are turning the pizza into many layered pizzas and turning all the cutting lines into cutting planes.
            
            $$\Delta S_n = P_n = \binom{n}{0} + \binom{n}{1} + \binom{n}{2} = 1 + \frac{n^{\underline{1}}}{1!} + \frac{n^{\underline{2}}}{2!} = 1 + n^{\underline{1}} + \frac{1}{2}n^{\underline{2}}$$

            $\Delta^{(1)} f(n) = 1 + n^{\underline{1}} + \frac{1}{2}n^{\underline{2}}$

            \begin{center}
                \begin{tabular}{l|ccccccccc}
                    {\bf $n$ } & 0 & 1 & 2 & 3 & 4 & 5 & 6 & 7 \\
                    \hline
                    {$\Delta^{(1)} f(n)$} & 1 & 2 & 4 & 7 & 11 & 16 & 22 & 29 \\
                    {$\Delta^{(2)} f(n)$} & 1 & 2 & 3 & 4 & 5 & 6 & 7    \\
                    {$\Delta^{(3)} f(n)$} & 1 & 1 & 1 & 1 & 1 & 1    \\
                    {$\Delta^{(4)} f(n)$} & 0 & 0 & 0 & 0 & 0     \\
                    {$\Delta^{(5)} f(n)$} & 0 & 0 & 0 & 0     \\
                    {$\Delta^{(6)} f(n)$} & 0 & 0 & 0     \\
                \end{tabular}
            \end{center}

            $$ S_n = \sum_{k=0}^n (1 + n^{\underline{1}} + \frac{1}{2}n^{\underline{2}}) = \sum_{k=0}^n 1 + \sum_{k=0}^n k^{\underline{1}} + \frac{1}{2} \sum_{k=0}^n k^{\underline{2}}$$
            
            $$=(n+1) + [\ds\left .\frac{k^{\underline{2}}}{2}\right|_{k=0}^{k=n+1}] + \frac{1}{2} [\ds\left .\frac{k^{\underline{3}}}{3}\right|_{k=0}^{k=n+1}] $$

            $$= n+1 + \frac{1}{2}(n+1)^{\underline{2}} + \frac{1}{6}(n+1)^{\underline{3}}$$

            $$= n+1 +\frac{1}{2}(n+1)n + \frac{1}{6}(n+1)n(n-1) $$

            $$= \frac{1}{6}n^3 + \frac{1}{2}n^2 + \frac{4}{3}n + 1$$
            
            There you have it
            
            $$ S_n = \frac{1}{6}n^3 + \frac{1}{2}n^2 + \frac{4}{3}n + 1 $$
            
            Let $f(n) = S_n$

            \begin{center}
                \begin{tabular}{l|cccccccccc}
                    {\bf $n$ } & -1 & 0 & 1 & 2 & 3 & 4 & 5 & 6 & 7 \\
                    \hline
                    {$f(n)$} & 0 & 1 & 3 & 7 & 14 & 25 & 41 & 63 & 92 \\
                    {$\Delta^{(1)} f(n)$} & 1 & 2 & 4 & 7 & 11 & 16 & 22 & 29 \\
                    {$\Delta^{(2)} f(n)$} & 1 & 2 & 3 & 4 & 5 & 6 & 7    \\
                    {$\Delta^{(3)} f(n)$} & 1 & 1 & 1 & 1 & 1 & 1    \\
                    {$\Delta^{(4)} f(n)$} & 0 & 0 & 0 & 0 & 0     \\
                    {$\Delta^{(5)} f(n)$} & 0 & 0 & 0 & 0     \\
                    {$\Delta^{(6)} f(n)$} & 0 & 0 & 0     \\
                \end{tabular}
            \end{center}


            Notice the first table is in the second table just shifted down and to the left.
            
        \newpage
		\begin{enumerate}[resume]
		\item Use the Discrete Taylor Theorem to determine a closed formula for the sum $\ds \sum_{k=0}^n k^3$. 
        \end{enumerate}
        
            Given
            $$f(n)=\sum_{k \geq 0}\Delta^{(k)}f(a)\ds\binom{n-a}{k}$$
            $$\ds\binom{n}{k} = \frac{n^{\underline{k}}}{k!}$$
            For k not equal to -1
            $$\ds\sum_{n=a}^b n^{\underline{k}} = \ds\left .\frac{n^{\underline{k+1}}}{k+1}\right|_{n=a}^{n=b+1}$$
            
            Start
            
            Let $a=0$
            $$f(n)=\sum_{k \geq 0}\Delta^{(k)}f(0)\ds\binom{n}{k}$$
            $$k^3=0\ds\binom{k}{0}+1\ds\binom{k}{1}+6\ds\binom{k}{2}+6\ds\binom{k}{3}+0\ds\binom{k}{4}+0\ds\binom{k}{5}+0\ds\binom{k}{6}+0\ds\binom{k}{7}+...$$
            $$=1\frac{k^{\underline{1}}}{1!}+6\frac{k^{\underline{2}}}{2!}+6\frac{k^{\underline{3}}}{3!}$$
            $$=k^{\underline{1}}+3k^{\underline{2}}+k^{\underline{3}}$$
    
            $$ \ds\sum_{k=0}^n k^3 = \ds\sum_{k=0}^n (k^{\underline{1}}+3k^{\underline{2}}+k^{\underline{3}})$$
            $$=\ds\sum_{k=0}^n k^{\underline{1}}+3\ds\sum_{k=0}^n k^{\underline{2}}+\ds\sum_{k=0}^n k^{\underline{3}}$$
            $$=\ds\left .\frac{k^{\underline{2}}}{2}\right|_{k=0}^{k=n+1} + 3(\ds\left .\frac{k^{\underline{3}}}{3}\right|_{k=0}^{k=n+1}) + \ds\left .\frac{k^{\underline{4}}}{4}\right|_{k=0}^{k=n+1} $$
            $$=\frac{(n+1)n}{2} + (n+1)n(n-1) + \frac{(n+1)n(n-1)(n-2)}{4}$$
            $$=\frac{n^4}{4} + \frac{n^3}{2} + \frac{n^2}{4}$$
    
            There you have it
            $$\ds \sum_{k=0}^n k^3 = \frac{n^4}{4} + \frac{n^3}{2} + \frac{n^2}{4}$$
    
            End
        
        \newpage
        \begin{enumerate}[resume]
		\item 
            \begin{enumerate} 
            \item Writing the result in terms of binomial coefficients, use the Discrete Taylor Theorem to determine a closed formula for the sum $\sum_{k = -3}^n k^3(k+1)(k+2)(k+3)$ using center equal to $0$.  
            \end{enumerate}

                Given
                $$f(n)=\sum_{k \geq 0}\Delta^{(k)}f(a)\ds\binom{n-a}{k}$$
                Let $a=0$ as the prompt instructs

                Start
                
                $$k^3(k+1)(k+2)(k+3) = 0\ds\binom{k}{0} + 24\ds\binom{k}{1} + 432\ds\binom{k}{2} + 1872\ds\binom{k}{3} + 3264\ds\binom{k}{4} + 2520\ds\binom{k}{5} + 720\ds\binom{k}{6} + 0\ds\binom{k}{7} + 0\ds\binom{k}{8} + ...$$
                $$=24\frac{k^{\underline{1}}}{1!} + 432\frac{k^{\underline{2}}}{2!} + 1872\frac{k^{\underline{3}}}{3!} + 3264\frac{k^{\underline{4}}}{4!} + 2520\frac{k^{\underline{5}}}{5!} + 720\frac{k^{\underline{6}}}{6!}$$
                $$= 24 k^{\underline{1}} + 432 \frac{k^{\underline{2}}}{2} + 1872 \frac{k^{\underline{3}}}{6} + 3264 \frac{k^{\underline{4}}}{24} + 2520 \frac{k^{\underline{5}}}{120} + 720 \frac{k^{\underline{6}}}{720} $$
                $$= 24k^{\underline{1}} + 216k^{\underline{2}} + 312k^{\underline{3}} + 136k^{\underline{4}} + 21k^{\underline{5}} + k^{\underline{6}} $$

                Plugging in

                $$\sum_{k = -3}^n k^3(k+1)(k+2)(k+3) = \sum_{k = -3}^n (24k^{\underline{1}} + 216k^{\underline{2}} + 312k^{\underline{3}} + 136k^{\underline{4}} + 21k^{\underline{5}} + k^{\underline{6}}) $$
                $$= 24 \sum_{k = -3}^n k^{\underline{1}} + 216 \sum_{k = -3}^n k^{\underline{2}} + 312 \sum_{k = -3}^n k^{\underline{3}} + 136 \sum_{k = -3}^n k^{\underline{4}} + 21 \sum_{k = -3}^n k^{\underline{5}} + \sum_{k = -3}^n k^{\underline{6}}$$
                
                Using: for k not equal to -1 $\ds\sum_{n=a}^b n^{\underline{k}} = \ds\left .\frac{n^{\underline{k+1}}}{k+1}\right|_{n=a}^{n=b+1}$

                $$= 24(\ds\left .\frac{k^{\underline{2}}}{2}\right|_{k=-3}^{k=n+1}) + 216(\ds\left .\frac{k^{\underline{3}}}{3}\right|_{k=-3}^{k=n+1}) + 312(\ds\left .\frac{k^{\underline{4}}}{4}\right|_{k=-3}^{k=n+1}) + 136(\ds\left .\frac{k^{\underline{5}}}{5}\right|_{k=-3}^{k=n+1}) + 21(\ds\left .\frac{k^{\underline{6}}}{6}\right|_{k=-3}^{k=n+1}) + (\ds\left .\frac{k^{\underline{7}}}{7}\right|_{k=-3}^{k=n+1}) $$

                \begin{multline*}
                    = 24(\frac{(n+1)^{\underline{2}}}{2} - \frac{(-3)^{\underline{2}}}{2}) + 216(\frac{(n+1)^{\underline{3}}}{3} - \frac{(-3)^{\underline{3}}}{3}) + 312(\frac{(n+1)^{\underline{4}}}{4} - \frac{(-3)^{\underline{4}}}{4}) + \\ 136(\frac{(n+1)^{\underline{5}}}{5} - \frac{(-3)^{\underline{5}}}{5}) + 21(\frac{(n+1)^{\underline{6}}}{6} - \frac{(-3)^{\underline{6}}}{6}) + (\frac{(n+1)^{\underline{7}}}{7} - \frac{(-3)^{\underline{7}}}{7})
                \end{multline*}

                \begin{multline*}
                    = 12[(n+1)n - (-3)(-4)] + 72[(n+1)n(n-1) - (-3)(-4)(-5)] + \\ 78[(n+1)n(n-1)(n-2) - (-3)(-4)(-5)(-6)] + \frac{136}{5}[(n+1)n(n-1)(n-2)(n-3) - (-3)(-4)(-5)(-6)(-7)] + \\ \frac{21}{6}[(n+1)n(n-1)(n-2)(n-3)(n-4) - (-3)(-4)(-5)(-6)(-7)(-8)] + \\ \frac{1}{7}[(n+1)n(n-1)(n-2)(n-3)(n-4)(n-5) - (-3)(-4)(-5)(-6)(-7)(-8)(-9)]
                \end{multline*}
                
                \begin{multline*}
                    = [72n^3+12n^2-60n+4176] + \\ [78n^4-156n^3-78n^2+156n-28080] + [\frac{136}{5}n^5-136n^4+136n^3+136n^2-\frac{816}{5}n+68544] + \\ [\frac{7}{2}n^6-\frac{63}{2}n^5+\frac{175}{2}n^4-\frac{105}{2}n^3-91n^2+84n-70560] + \\ [\frac{1}{7}n^7-2n^6+10n^5-20n^4+7n^3+22n^2-\frac{120}{7}n+25920]
                \end{multline*}

                \begin{multline*}
                    = 78n^4-84n^3-66n^2+96n-23904 + \frac{7}{2}n^6-\frac{43}{10}n^5-\frac{97}{2}n^4+\frac{167}{2}n^3+45n^2-\frac{396}{5}n-2016 + \\ \frac{1}{7}n^7-2n^6+10n^5-20n^4+7n^3+22n^2-\frac{120}{7}n+25920
                \end{multline*}

                $$=\frac{1}{7}n^7+\frac{3}{2}n^6+\frac{57}{10}n^5+\frac{19}{2}n^4+\frac{13}{2}n^3+n^2-\frac{12}{35}n$$

                There you have it

                $$\sum_{k = -3}^n k^3(k+1)(k+2)(k+3)=\frac{1}{7}n^7+\frac{3}{2}n^6+\frac{57}{10}n^5+\frac{19}{2}n^4+\frac{13}{2}n^3+n^2-\frac{12}{35}n$$

                End
    
            \newpage
            \begin{enumerate}[resume]
            \item Repeat part (a) but with center equal to $-4$. 
            \end{enumerate}

                Restate the problem: 
                Writing the result in terms of binomial coefficients, use the Discrete Taylor Theorem to determine a closed formula for the sum $\sum_{k = -3}^n k^3(k+1)(k+2)(k+3)$ using center equal to $-4$.

                Given
                $$f(n)=\sum_{k \geq 0}\Delta^{(k)}f(a)\ds\binom{n-a}{k}$$
                Let $a=-4$ as the prompt instructs
                $$f(n)=\sum_{k \geq 0}\Delta^{(k)}f(-4)\ds\binom{n+4}{k}$$

                Start

                \begin{multline*}
                    k^3(k+1)(k+2)(k+3) = 384\ds\binom{(k+4)}{0} -384\ds\binom{(k+4)}{1} + 384\ds\binom{(k+4)}{2} - \\ 384\ds\binom{(k+4)}{3} + 384\ds\binom{(k+4)}{4} -360\ds\binom{(k+4)}{5} + 720\ds\binom{(k+4)}{6} + 0\ds\binom{(k+4)}{7} + 0\ds\binom{(k+4)}{8} + ...
                \end{multline*}
                $$= 384 -384\frac{(k+4)^{\underline{1}}}{1!} + 384\frac{(k+4)^{\underline{2}}}{2!} -384\frac{(k+4)^{\underline{3}}}{3!} + 384\frac{(k+4)^{\underline{4}}}{4!} -360\frac{(k+4)^{\underline{5}}}{5!} + 720\frac{(k+4)^{\underline{6}}}{6!}$$
                
                $$= 384 -384(k+4)^{\underline{1}} + 192(k+4)^{\underline{2}} -64(k+4)^{\underline{3}} + 16(k+4)^{\underline{4}} -3(k+4)^{\underline{5}} + (k+4)^{\underline{6}}$$

                Plugging in

                \begin{multline*}
                    \sum_{k = -3}^n k^3(k+1)(k+2)(k+3) = \\ \sum_{k = -3}^n (384 -384(k+4)^{\underline{1}} + 192(k+4)^{\underline{2}} -64(k+4)^{\underline{3}} + 16(k+4)^{\underline{4}} -3(k+4)^{\underline{5}} + (k+4)^{\underline{6}})
                \end{multline*}

                \begin{multline*}
                    = 384 \sum_{k = -3}^n 1 -384 \sum_{k = -3}^n (k+4)^{\underline{1}} + 192 \sum_{k = -3}^n (k+4)^{\underline{2}} -64 \sum_{k = -3}^n (k+4)^{\underline{3}} + 16 \sum_{k = -3}^n (k+4)^{\underline{4}} -3 \sum_{k = -3}^n (k+4)^{\underline{5}} + \sum_{k = -3}^n (k+4)^{\underline{6}}
                \end{multline*}

                \begin{multline*}
                    = 384(n+4) - 384 (\ds\left .\frac{(k+4)^{\underline{2}}}{2}\right|_{k=-3}^{k=n+1}) + 192(\ds\left .\frac{(k+4)^{\underline{3}}}{3}\right|_{k=-3}^{k=n+1}) - 64(\ds\left .\frac{(k+4)^{\underline{4}}}{4}\right|_{k=-3}^{k=n+1}) + \\ 16 (\ds\left .\frac{(k+4)^{\underline{5}}}{5}\right|_{k=-3}^{k=n+1}) -3(\ds\left .\frac{(k+4)^{\underline{6}}}{6}\right|_{k=-3}^{k=n+1}) + (\ds\left .\frac{(k+4)^{\underline{7}}}{7}\right|_{k=-3}^{k=n+1})
                \end{multline*}

                \begin{multline*}
                    = 384[n+4] - 384[\frac{((n+1)+4)^{\underline{2}}}{2} - \frac{((-3)+4)^{\underline{2}}}{2}] + 192[\frac{((n+1)+4)^{\underline{3}}}{3} - \frac{((-3)+4)^{\underline{3}}}{3}] - \\ 64[\frac{((n+1)+4)^{\underline{4}}}{4} - \frac{((-3)+4)^{\underline{4}}}{4}] + 16[\frac{((n+1)+4)^{\underline{5}}}{5} - \frac{((-3)+4)^{\underline{5}}}{5}] - \\ 3[\frac{((n+1)+4)^{\underline{6}}}{6} - \frac{((-3)+4)^{\underline{6}}}{6}] + [\frac{((n+1)+4)^{\underline{7}}}{7} - \frac{((-3)+4)^{\underline{7}}}{7}]
                \end{multline*}

                \begin{multline*}
                    = 384n+1536 - 192[(n+5)^{\underline{2}} - (1)^{\underline{2}}] + 64[(n+5)^{\underline{3}} - (1)^{\underline{3}}] - \\ 16[(n+5)^{\underline{4}} - (1)^{\underline{4}}] + \frac{16}{5}[(n+5)^{\underline{5}} - (1)^{\underline{5}}] - \frac{1}{2}[(n+5)^{\underline{6}} - (1)^{\underline{6}}] + \frac{1}{7}[(n+5)^{\underline{7}} - (1)^{\underline{7}}]
                \end{multline*}

                \begin{multline*}
                    = 384n+1536 - 192(n+5)(n+4) + 64(n+5)(n+4)(n+3) - \\ 16(n+5)(n+4)(n+3)(n+2) + \frac{16}{5}(n+5)(n+4)(n+3)(n+2)(n+1) - \\ \frac{1}{2}(n+5)(n+4)(n+3)(n+2)(n+1)n + \frac{1}{7}(n+5)(n+4)(n+3)(n+2)(n+1)n(n-1)
                \end{multline*}

                $$ = \frac{1}{7}n^7+\frac{3}{2}n^6+\frac{57}{10}n^5+\frac{19}{2}n^4+\frac{13}{2}n^3+n^2-\frac{12}{35}n$$

                There you have it

                $$\sum_{k = -3}^n k^3(k+1)(k+2)(k+3) = \frac{1}{7}n^7+\frac{3}{2}n^6+\frac{57}{10}n^5+\frac{19}{2}n^4+\frac{13}{2}n^3+n^2-\frac{12}{35}n$$
                
                End
            
            \newpage
            \begin{enumerate}[resume]
            \item The formulae from (a) and (b) should look different, but are they?  Determine whether they are the same or not.
    	\end{enumerate}

                The formula do look different but the solution is the same.

                Formula (a)
                $$\sum_{k = -3}^n k^3(k+1)(k+2)(k+3) = \sum_{k = -3}^n (24k^{\underline{1}} + 216k^{\underline{2}} + 312k^{\underline{3}} + 136k^{\underline{4}} + 21k^{\underline{5}} + k^{\underline{6}}) $$
                
                Formula (b)
                \begin{multline*}
                    \sum_{k = -3}^n k^3(k+1)(k+2)(k+3) = \\ \sum_{k = -3}^n (384 -384(k+4)^{\underline{1}} + 192(k+4)^{\underline{2}} -64(k+4)^{\underline{3}} + 16(k+4)^{\underline{4}} -3(k+4)^{\underline{5}} + (k+4)^{\underline{6}})
                \end{multline*}

                So
                \begin{multline*}
                    \sum_{k = -3}^n (24k^{\underline{1}} + 216k^{\underline{2}} + 312k^{\underline{3}} + 136k^{\underline{4}} + 21k^{\underline{5}} + k^{\underline{6}}) = \\ \sum_{k = -3}^n (384 -384(k+4)^{\underline{1}} + 192(k+4)^{\underline{2}} -64(k+4)^{\underline{3}} + 16(k+4)^{\underline{4}} -3(k+4)^{\underline{5}} + (k+4)^{\underline{6}})
                \end{multline*}
                
                They both come out to $\frac{1}{7}n^7+\frac{3}{2}n^6+\frac{57}{10}n^5+\frac{19}{2}n^4+\frac{13}{2}n^3+n^2-\frac{12}{35}n$ so they are the same.

                Why are they the same though? I notice they both have the same number of terms at a falling power and at the same falling powers (i.e. 1, 2, 3, 4, 5, 6) with falling 6 being the highest falling power. Moving the center of the discrete Taylor theorem means we grab different coefficients for the binomial coefficients. These coefficients much make up for the fact that we shifted the center. This kinda makes since as the discrete Taylor theorem says that if you know all the derivatives of a function then you can rebuild the function. Where you start building from doesn't matter as long as you are using data from the same derivative table and keep account of the shift taken.
      
        \end{enumerate}

        \newpage
        \begin{enumerate}[resume]
	\item Use summation by parts to find a closed formula for $\ds \sum_{k=0}^n\frac{H_k}{(k+1)(k+2)}$.
	\end{enumerate}

            Given
                $$\sum_{k=0}^n u(k) \Delta v(k) = \ds\left . u(k)v(k) \right|_{k=0}^{k=n+1}  - \sum_{k=0}^n Ev(k) \Delta u(k)$$

                
            Restate Problem
                $$\ds \sum_{k=0}^n H_k \frac{1}{(k+1)(k+2)} = \ds \sum_{k=0}^n H_k k^{\underline{-2}} $$
    
            Let 
                $$ u(k) = H_k $$ 
                $$ \Delta u(k) = \frac{1}{k+1} $$
                
                $$ v(k) = \sum_{i=0}^{k-1} i^{\underline{-2}} = \ds\left .\frac{i^{\underline{-1}}}{-1}\right|_{i=0}^{i=k} = -k^{\underline{-1}} + 0^{\underline{-1}} = 1 - \frac{1}{k+1}$$
                
                $$ \Delta v(k) = k^{\underline{-2}} = \Delta (1 - \frac{1}{k+1}) = \Delta 1 - \Delta \frac{1}{k+1}) = 0 - (\frac{1}{k+2} - \frac{1}{k+1}) = \frac{1}{k+1} - \frac{1}{k+2} = \frac{1}{(k+1)(k+2)} $$

            There for
                $$\sum_{k=0}^n H_k k^{\underline{-2}} = \ds\left . H_k (1-\frac{1}{k+1}) \right|_{k=0}^{k=n+1}  - \sum_{k=0}^n (E(1-\frac{1}{k+1})) \frac{1}{k+1} $$
                
                $$=H_{n+1}(1-\frac{1}{n+2})-\sum_{k=0}^n (E(1-\frac{1}{k+1})) \frac{1}{k+1}$$

                $$ = H_{n+1}-\frac{H_{n+1}}{n+2}-\sum_{k=0}^n (1-\frac{1}{k+2}) \frac{1}{k+1} $$

                $$ = H_{n+1}-\frac{H_{n+1}}{n+2}-\sum_{k=0}^n (\frac{1}{k+1}-\frac{1}{(k+1)(k+2)}) $$

                $$ = H_{n+1}-\frac{H_{n+1}}{n+2}-(\sum_{k=0}^n \frac{1}{k+1}-\sum_{k=0}^n\frac{1}{(k+1)(k+2)}) $$

                $$ = H_{n+1}-\frac{H_{n+1}}{n+2}-\sum_{k=0}^n k^{\underline{-1}} + \sum_{k=0}^n k^{\underline{-2}} $$

                $$ = H_{n+1}-\frac{H_{n+1}}{n+2} - (\ds\left .H_k\right|_{k=0}^{k=n+1}) + (\ds\left .\frac{k^{\underline{-1}}}{-1}\right|_{k=0}^{k=n+1}) $$

                $$ = H_{n+1}-\frac{H_{n+1}}{n+2} - (H_{n+1} - H_0) + (\frac{(n+1)^{\underline{-1}}}{-1} - \frac{0^{\underline{-1}}}{-1}) $$

                $$ = - \frac{H_{n+1}}{n+2} + H_0 - \frac{1}{n+2} + 1 $$

                $$ = - \frac{H_{n+1}-1}{n+2} + H_0 + 1 $$

            There you go
                $$ \sum_{k=0}^n\frac{H_k}{(k+1)(k+2)} = - \frac{H_{n+1}-1}{n+2} + H_0 + 1 $$
            I don't know what $H_0$ would mean though as $\sum_{k=1}^n \frac{1}{n}$ is the definition for $H_n$.

	\noindent \underline{\hspace{3in}}\\
	
	
	
	
\end{document}