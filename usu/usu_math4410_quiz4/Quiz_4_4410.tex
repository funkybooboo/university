\documentclass[10pt, AMS Euler]{article}
\textheight=9.25in \textwidth=7in \topmargin=-.75in
\oddsidemargin=-0.25in
\evensidemargin=-0.25in
\usepackage{url}  % The bib file uses this
\usepackage{graphicx} %to import pictures
\usepackage{amsmath, amssymb, color, wasysym}
\usepackage{theorem, concrete, multicol,tikz}
\usepackage[normalem]{ulem} %for strikethrough (\sout{blah})
\usepackage{enumitem}

\setlength{\intextsep}{5mm} \setlength{\textfloatsep}{5mm}
\setlength{\floatsep}{5mm}


{\theorembodyfont{\rmfamily}
	\newtheorem{definition}{Definition}[section]}
{\theorembodyfont{\rmfamily} \newtheorem{example}{Example}[section]}
{\theorembodyfont{\rmfamily} \newtheorem{lemma}{Lemma}[section]}
{\theorembodyfont{\rmfamily} \newtheorem{theorem}{Theorem}[section]}
{\theorembodyfont{\rmfamily} \newenvironment{proof}{\par{\it
			Proof:}}{\nopagebreak[4]\rule{2mm}{2mm}}}
{\theorembodyfont{\rmfamily}
	\newenvironment{solution}{\par{\bf{Solution:}}}{\nopagebreak[4]\rule{2mm}{2mm}}}

\usetikzlibrary{arrows}
\usetikzlibrary{shapes}
\newcommand{\mymk}[1]{%
	\tikz[baseline=(char.base)]\node[anchor=south west, draw,rectangle, rounded corners, inner sep=2pt, minimum size=7mm,
	text height=2mm](char){\ensuremath{#1}} ;}

\newcommand*\circled[1]{\tikz[baseline=(char.base)]{
		\node[shape=circle,draw,inner sep=2pt] (char) {#1};}}


%%%%  SHORTCUT COMMANDS  %%%%
\newcommand{\ds}{\displaystyle}
\newcommand{\Z}{\mathbb{Z}}
\newcommand{\arc}{\rightarrow}
\newcommand{\R}{\mathbb{R}}
\newcommand{\N}{\mathbb{N}}
\newcommand{\Q}{\mathbb{Q}}
\newcommand{\stirling}[2]{\genfrac{\{}{\}}{0pt}{}{#1}{#2}}

%%%%  footnote style %%%%

\renewcommand{\thefootnote}{\fnsymbol{footnote}}

\pagestyle{empty}
\begin{document}
	
	\noindent{\bf \large MATH 4410 }\\

        Nate Stott A02386053 \\
	
	\noindent \underline{\hspace{2in}}\\
	
	{\bf Quiz \#4; Due 11:59 pm, 2/2/2024}\\

        \newpage
	\begin{enumerate}
		\item Construct a (the) generating function for the sequence of odd positive integers; that is, construct a function $f(x)$ such that $f(x) = \sum_{n \geq 0}(2n+1)x^n$.
        \end{enumerate}
        Given 
            $$ \sum_{n \geq 0} x^n = \frac{1}{1-x} = 1 + x + x^2 + x^3 + x^4 + x^5 + ... $$
            $$ \sum_{n \geq 0} n x^n = \frac{x}{(1-x)^2}$$

        Start
        $$\sum_{n \geq 0}(2n+1)x^n = 1 + 3x + 5x^2 + 7x^3 + 9x^4 + 11x^5 + ... $$
        $$ = \sum_{n \geq 0}(2nx^n+x^n) $$
        $$ = \sum_{n \geq 0} 2nx^n + \sum_{n \geq 0} x^n $$
        $$ = 2 \sum_{n \geq 0} nx^n + \frac{1}{1-x} $$
        $$ = \frac{2x}{(1-x)^2} + \frac{1}{1-x} $$
        $$ = \frac{2x}{(1-x)^2} + \frac{1-x}{(1-x)^2} $$
        $$ = \frac{2x+1-x}{(1-x)^2} $$
        $$ = \frac{x+1}{(1-x)^2} $$
        End
        $$ \sum_{n \geq 0}(2n+1)x^n = \frac{x+1}{(1-x)^2} $$

        \newpage
	\begin{enumerate}[resume]
		\item Construct a (the) generating function $g(x)$ for the sequence $(a_n)_{n \geq 0}$, where $a_n = \ds \sum_{i = 0}^n(2i+1)$. 
        \end{enumerate}

        Given 
        $$ \sum_{n \geq 0} x^n = \frac{1}{1-x} $$
        $$ \sum_{n \geq 0} n x^n = \frac{x}{(1-x)^2} $$

        Start
        $$ g(x) = \sum_{n\geq0} ( \sum_{i=0}^n (2i+1) ) x^n = 1x^0 + 4x^1 + 9x^2 + 16x^3 + ... $$
        $$ \sum_{i=0}^n (2i+1) = (n+1)^2 $$
        $$ \sum_{n\geq0} ( \sum_{i=0}^n (2i+1) ) x^n = \sum_{n\geq0} (n+1)^2 x^n $$
        $$ = \sum_{n\geq0} (n^2+2n+1) x^n $$
        $$ = \sum_{n\geq0} n^2 x^n + 2 \sum_{n\geq0} n x^n + \sum_{n\geq0} x^n $$
        $$ = \sum_{n\geq0} n^2 x^n + \frac{2x}{(1-x)^2} + \frac{1}{1-x} $$
        $$ = \sum_{n\geq0} n^2 x^n + \frac{x+1}{(1-x)^2} $$
        Focus 
        $$ x^n = xx^{n-1} $$
        $$ x \frac{d}{dx} (x^n) = x n x^{n-1} = n x^n $$ 
        $$ x \frac{d}{dx} (x \frac{d}{dx} ( x^n ) ) = n^2 x^n $$
        Back to it
        $$ = \sum_{n\geq0} (x \frac{d}{dx} ( x \frac{d}{dx} (x^n))) + \frac{x+1}{(1-x)^2} $$
        $$ = x \frac{d}{dx} ( x \frac{d}{dx} ( \sum_{n\geq0} (x^n) )) + \frac{x+1}{(1-x)^2} $$
        $$ = x \frac{d}{dx} ( x \frac{d}{dx} ( \frac{1}{1-x} )) + \frac{x+1}{(1-x)^2} $$
        $$ = \frac{x^2 + x}{(1-x)^3} + \frac{x+1}{(1-x)^2} $$
        $$ = \frac{x^2 + x}{(1-x)^3} + \frac{x+1}{(1-x)^2} $$
        $$ = \frac{x+1}{(1-x)^3} $$
        End
        $$ g(x) = \sum_{n\geq0} ( \sum_{i=0}^n (2i+1) ) x^n = \frac{x+1}{(1-x)^3} $$

        
        \newpage
        \begin{enumerate}[resume]
		\item From $g(x)$ found above, extract a formula for the coefficient on $x^n$, and verify that it is indeed $\ds\sum_{i=0}^n(2i+1)$.
        \end{enumerate}

        Start
        $$ \frac{x+1}{(1-x)^3} = \frac{A}{(1-x)} + \frac{B}{(1-x)^2} + \frac{C}{(1-x)^3} $$
        I could write down the math but ...  $A = 0$, $B=-1$, $C=2$
        $$ \frac{x+1}{(1-x)^3} = \frac{-1}{(1-x)^2} + \frac{2}{(1-x)^3} $$
        $$ = - (\frac{1}{(1-x)^2}) + 2 (\frac{1}{(1-x)^3}) $$
        Using FGF2 to get back to the summations
        $$ \frac{1}{(1-x)^k} = \sum_{n\geq0} \binom{n+k-1}{n} x^n $$
        $$ = - ( \sum_{n\geq0} (1+n) x^n ) + 2 ( \sum_{n\geq0} \frac{1}{2} (1+n)(2+n) x^n ) $$
        $$ = - ( \sum_{n\geq0} (1+n) x^n ) + ( \sum_{n\geq0} (1+n)(2+n) x^n ) $$
        Here is the part where I turn the summations into functions that give me $a_n$
        $$ a_n = - (1+n) + (1+n)(2+n)  $$
        $$ a_n = n^2 + 2n + 1  $$
        End \\
        Checked for n = 0, 1, 2, 3

        \newpage
	\begin{enumerate}[resume]
		\item Construct a generating function for $(b_n)_{n \geq 0}$, where $b_n = \ds \sum_{i=0}^n i^2$, and obtain a closed formula for $b_n$ from the generating function. 
	\end{enumerate}

        Start
        $$ g(x) = \sum_{n\geq0} (\sum_{i=0}^n (i^2) x^n) $$
        $$ f(n) = \sum_{i=0}^n (i^2) $$
        $$ \Delta f(n) = (n+1)^2 $$
        Back to the Discrete Taylor Theorem
        $$f(n)=\sum_{k \geq 0}\Delta^{(k)}f(a)\ds\binom{n-a}{k}$$
        \begin{center}
                \begin{tabular}{l|ccccccccc}
                    {\bf $n$ } & 0 & 1 & 2 & 3 & 4 & 5 & 6 & \\
                    \hline
                    {$\Delta^{(0)} f(n)$} & 0 & 1 & 5 & 14 & 30 & 55 & 91 & \\
                    {$\Delta^{(1)} f(n)$} & 1 & 4 & 9 & 16 & 25 & 36 & \\
                    {$\Delta^{(2)} f(n)$} & 3 & 5 & 7 & 9 & 11 & \\
                    {$\Delta^{(3)} f(n)$} & 2 & 2 & 2 & 2 & \\
                    {$\Delta^{(4)} f(n)$} & 0 & 0 & 0 & \\
                \end{tabular}
            \end{center}
        Let $a=0$
        $$ f(n) = 0 \binom{n}{0} + 1 \binom{n}{1} + 3 \binom{n}{2} + 2 \binom{n}{3} + 0 \binom{n}{4} + ... $$
        $$ f(n) = 1 \binom{n}{1} + 3 \binom{n}{2} + 2 \binom{n}{3} $$
        Let use this $\binom{n}{k} = \frac{n^{\underline{k}}}{k!}$
        $$ f(n) = 1 \frac{n^{\underline{1}}}{1!} + 3 \frac{n^{\underline{2}}}{2!} + 2 \frac{n^{\underline{3}}}{3!} $$
        $$ f(n) = n + 3 \frac{n(n-1)}{2} + 2 \frac{n(n-1)(n-2)}{6} $$
        $$ f(n) = n + \frac{3n^2-3n}{2}  + \frac{(n^2-n)(n-2)}{3} $$
        $$ f(n) = n + \frac{3n^2-3n}{2}  + \frac{n^3-3n^2+2n}{3} $$
        $$ f(n) = \frac{6n}{6} + \frac{3(3n^2-3n)}{6}  + \frac{2(n^3-3n^2+2n)}{6} $$
        $$ f(n) = \frac{6n + 9n^2-9n + 2n^3-6n^2+4n}{6} $$
        $$ f(n) = \frac{2n^3 + 3n^2 + n}{6} $$
        So
        $$ f(n) = \sum_{i=0}^n (i^2) = \frac{2n^3 + 3n^2 + n}{6} $$
        Checked for n = 0, 1, 2, 3 \\
        Back to the point of all this,
        $$ g(x) = \sum_{n\geq0} ( ( \frac{2n^3 + 3n^2 + n}{6} ) x^n ) $$
        $$ g(x) = \frac{1}{6} \sum_{n\geq0} ( 2n^3 x^n + 3n^2 x^n + nx^n ) $$
        $$ g(x) = \frac{1}{6} ( 2 (\sum_{n\geq0} n^3 x^n ) + 3 ( \sum_{n\geq0} n^2 x^n ) + \sum_{n\geq0} nx^n ) $$
        I need to find
        $$ \sum_{n\geq0} n^3 x^n $$
        Focus
        $$ n^3 x^n = x \frac{d}{dx} (x \frac{d}{dx} (x \frac{d}{dx} (x^n)) ) $$
        So
        $$ \sum_{n\geq0} n^3 x^n = \sum_{n\geq0} ( x \frac{d}{dx} (x \frac{d}{dx} (x \frac{d}{dx} (x^n)) ) ) $$
        $$ \sum_{n\geq0} n^3 x^n = x \frac{d}{dx} (x \frac{d}{dx} (x \frac{d}{dx} ( \sum_{n\geq0} x^n)) ) $$
	$$ \sum_{n\geq0} n^3 x^n = \frac{(x^3+4x^2+x)}{(1-x)^4} $$
	Alright
        $$ \sum_{n \geq 0} x^n = \frac{1}{1-x} $$
        $$ \sum_{n \geq 0} n x^n = \frac{x}{(1-x)^2} $$
        $$ \sum_{n \geq 0} n^2 x^n = \frac{x^2 + x}{(1-x)^3} $$ 
        $$ \sum_{n\geq0} n^3 x^n = \frac{(x^3+4x^2+x)}{(1-x)^4} $$
        Back to the point
        $$ g(x) = \frac{1}{6} ( 2 (\frac{(x^3+4x^2+x)}{(1-x)^4} ) + 3 ( \frac{x^2 + x}{(1-x)^3} ) + \frac{x}{(1-x)^2} ) $$
        $$ g(x) = \frac{1}{6} ( 2 (\frac{(x^3+4x^2+x)}{(1-x)^4} ) + 3 ( \frac{(x^2 + x)(1-x)}{(1-x)^3(1-x)} ) + \frac{x(1-x)^2}{(1-x)^2(1-x)^2} ) $$
        $$ g(x) = \frac{1}{6} ( \frac{2x^3+8x^2+2x}{(1-x)^4} + \frac{3x-3x^3}{(1-x)^4} + \frac{x^3 - 2x^2 + x}{(1-x)^4} ) $$
        $$ g(x) = \frac{1}{6} ( \frac{2x^3+8x^2+2x + 3x-3x^3 + x^3-2x^2+x}{(1-x)^4} ) $$
        $$ g(x) = \frac{1}{6} ( \frac{ 6x^2 + 6x }{(1-x)^4} ) $$
        $$ g(x) = \frac{ x^2 + x }{(1-x)^4} $$
        Ok so the generating function is
        $$ g(x) = \frac{ x^2 + x }{(1-x)^4} $$
        Now it's time to get a closed formula for $b_n$ out of $g(x)$ \\
        Partial fractions!
        $$ g(x) = \frac{ x^2 + x }{(1-x)^4} = \frac{A}{(1-x)} + \frac{B}{(1-x)^2} + \frac{C}{(1-x)^3} + \frac{D}{(1-x)^4} $$
        Here are the results $A=0$, $B=1$, $C=-3$, $D=2$
        $$ g(x) = \frac{ x^2 + x }{(1-x)^4} = \frac{1}{(1-x)^2} - \frac{3}{(1-x)^3} + \frac{2}{(1-x)^4} $$
        $$ = \frac{1}{(1-x)^2} -3 (\frac{1}{(1-x)^3}) + 2 (\frac{1}{(1-x)^4}) $$
        Using FGF2 to get back to the summations,
        $$ \frac{1}{(1-x)^k} = \sum_{n\geq0} \binom{n+k-1}{n} x^n $$
        $$ = \sum_{n\geq0}((n+1)x^n) -3 (\sum_{n\geq0}(\frac{1}{2}(n+1)(n+2)x^n)) + 2 (\sum_{n\geq0}(\frac{1}{6}(n+1)(n+2)(n+3)x^n)) $$
        Here is the part where I drop the summations and $x^n$
        $$ b_n = (n+1) - \frac{3}{2}(n+1)(n+2) + \frac{1}{3}(n+1)(n+2)(n+3) $$
        $$ b_n = \frac{n^3}{3} + \frac{n^2}{2} + \frac{n}{6} $$
        And there is the closed form for $b_n$
        $$ b_n = \frac{n^3}{3} + \frac{n^2}{2} + \frac{n}{6} $$
        Checked for n = 0, 1, 2, 3
        
        
 
	\noindent \underline{\hspace{3in}}\\
	
	
	
	
\end{document}