\documentclass[10pt, AMS Euler]{article}
\textheight=9.25in \textwidth=7in \topmargin=-.75in
\oddsidemargin=-0.25in
\evensidemargin=-0.25in
\usepackage{url}  % The bib file uses this
\usepackage{graphicx} %to import pictures
\usepackage{amsmath, amssymb, color, wasysym}
\usepackage{theorem, concrete, multicol,tikz}
\usepackage[normalem]{ulem} %for strikethrough (\sout{blah})


\setlength{\intextsep}{5mm} \setlength{\textfloatsep}{5mm}
\setlength{\floatsep}{5mm}


{\theorembodyfont{\rmfamily}
	\newtheorem{definition}{Definition}[section]}
{\theorembodyfont{\rmfamily} \newtheorem{example}{Example}[section]}
{\theorembodyfont{\rmfamily} \newtheorem{lemma}{Lemma}[section]}
{\theorembodyfont{\rmfamily} \newtheorem{theorem}{Theorem}[section]}
{\theorembodyfont{\rmfamily} \newenvironment{proof}{\par{\it
			Proof:}}{\nopagebreak[4]\rule{2mm}{2mm}}}
{\theorembodyfont{\rmfamily}
	\newenvironment{solution}{\par{\bf{Solution:}}}{\nopagebreak[4]\rule{2mm}{2mm}}}

\usetikzlibrary{arrows}
\usetikzlibrary{shapes}
\newcommand{\mymk}[1]{%
	\tikz[baseline=(char.base)]\node[anchor=south west, draw,rectangle, rounded corners, inner sep=2pt, minimum size=7mm,
	text height=2mm](char){\ensuremath{#1}} ;}

\newcommand*\circled[1]{\tikz[baseline=(char.base)]{
		\node[shape=circle,draw,inner sep=2pt] (char) {#1};}}


%%%%  SHORTCUT COMMANDS  %%%%
\newcommand{\ds}{\displaystyle}
\newcommand{\Z}{\mathbb{Z}}
\newcommand{\arc}{\rightarrow}
\newcommand{\R}{\mathbb{R}}
\newcommand{\N}{\mathbb{N}}
\newcommand{\Q}{\mathbb{Q}}
\newcommand{\primes}{\mathscr{P}}


%%%%  footnote style %%%%

\renewcommand{\thefootnote}{\fnsymbol{footnote}}

\pagestyle{empty}
\begin{document}
	
	\noindent{\bf \large MATH 4410 }\\

    \noindent{\bf \large Nate Stott A02386053 }\\
	
	\noindent \underline{\hspace{2in}}\\
	
	\noindent {\bf Quiz \#8; Due 11:59 pm, 3/25/2024}\\
	
	
	
	
	\begin{enumerate}

 
	    \newpage	
		\item Recall: the \emph{Euler totient function} $\varphi(n)$ is defined to be the number of positive integers in the interval $[1..n]$ that are relatively prime to $n$. 
  
		Use the Principle of Inclusion/Exclusion to prove 
  
		$$ \varphi(n) = n \prod_{p | n}\left(1-\frac{1}{p}\right) $$

        Well lets play around

        Let $n=10$

        So the interval would be 

        $1$ $2$ $3$ $4$ $5$ $6$ $7$ $8$ $9$ $10$

        Relatively prime means that they share no common factor other than $1$.

        $1$ and $10$ are relatively prime

        $2$ and $10$ are not relatively prime as they share a factor of $2$

        $3$ and $10$ are relatively prime

        $4$ and $10$ are not relatively prime

        $5$ and $10$ are not relatively prime

        $6$ and $10$ are not relatively prime

        $7$ and $10$ are relatively prime

        $8$ and $10$ are not relatively prime

        $9$ and $10$ are relatively prime

        $10$ and $10$ are not relatively prime

        So $\varphi(n)$ should be equal to $4$

        Let's try it out

        $$ \varphi(n) = n \prod_{p | n}\left(1-\frac{1}{p}\right) $$
        $$ \varphi(10) = 10 \prod_{p | 10}\left(1-\frac{1}{p}\right) $$
        $$ \varphi(10) = 10 (1-\frac{1}{2})(1-\frac{1}{5}) $$
        $$ \varphi(10) = 10 (\frac{1}{2})(\frac{4}{5}) $$
        $$ \varphi(10) = 5(\frac{4}{5}) $$
        $$ \varphi(10) = 4 $$

        What about for an odd number?

        Let $n = 9$
        
        $1$ and $9$ are relatively prime

        $2$ and $9$ are relatively prime

        $3$ and $9$ are not relatively prime

        $4$ and $9$ are relatively prime

        $5$ and $9$ are relatively prime

        $6$ and $9$ are not relatively prime

        $7$ and $9$ are relatively prime

        $8$ and $9$ are relatively prime

        $9$ and $9$ are not relatively prime

        So $\varphi(9)$ should be $6$

        $$ \varphi(n) = n \prod_{p | n}\left(1-\frac{1}{p}\right) $$
        $$ \varphi(9) = 9 \prod_{p | 9}\left(1-\frac{1}{p}\right) $$
        $$ \varphi(9) = 9 (1-\frac{1}{3}) $$
        $$ \varphi(9) = 9 (\frac{2}{3}) $$
        $$ \varphi(9) = 6 $$
        

        Well, both those worked, so let's break what happened apart to try to understand why it worked.

        If n is even

        i. there will be less relatively prime numbers because you have to take out half of the numbers in the interval at least.
        
        ii. there will be $\frac{n}{2}$ in the totient function, so $\varphi(n)$ will be at least half of $n$.
        
        If n is odd

        i. There are more relatively prime numbers in the interval

        
        Looking at the equations again
        
		$$ \varphi(10) = 10 (1-\frac{1}{2})(1-\frac{1}{5}) $$
        Foil
        $$ \varphi(10) = 10 - \frac{10}{2} - \frac{10}{5} + \frac{1}{10} $$


        $$ \varphi(9) = 9 (1-\frac{1}{3}) $$
        Foil
        $$ \varphi(9) = 9-\frac{9}{3} $$


        $n/m$ can be thought of the number of divisors of n relative to m.

        An example, $1000/10$ there are $10$ divisors of $1000$ that can also be divided by 10. Namely, 100, 200, 300, 400, 500, 600, 700, 800, 900, 1000.

        So what we are saying in this $ \varphi(9) = 9-\frac{9}{3} $ is that the number of positive integers in the interval $[1..9]$ that are relatively prime to $9$ is $9$ but take away the number of divisors of 9 relative to 3 and that is 3 (3, 6, and 9). 

        This makes sense, as you need to take away numbers that can be divisible by another number other than 1.
        

        $$ \varphi(10) = 10 - \frac{10}{2} - \frac{10}{5} + \frac{1}{10} $$

         This is saying we need to take away all the numbers that are divisible by 2 and also 5. Then we need to add the numbers that are only divisible by 10 because we subtracted it off once too many times. 


        Generally
        
        $$ \varphi(n) = n \prod_{p | n}\left(1-\frac{1}{p}\right) = n (1-\frac{1}{p_1}-\frac{1}{p_2}-\frac{1}{p_3}-...+\frac{1}{p_1 p_2 p_3 ...}) $$

        So take away all the numbers that have other divisors, then 1. Then add back the numbers that got subtracted away too many times. 


        \newpage	
		\item Compute  $864^{-1} \!\! \mod 899$; that is, determine $x$ such that $864 \cdot x \equiv 1 \!\!\mod 899$.

        $$ 864 = 2^5 * 3^3 $$
        
        So 
        
        $$ 864^{-1} = (2^5 * 3^3)^{-1} = (2^{-1})^5 * (3^{-1})^3 $$
        
        I made a program that will give me the inverse of a number
        
-------------------------------------------------------------------------- \\
            \# mod n \\
            n = 899 \\
            
            \# limit l \\
            l = 900 \\
            
            \# array a \\
            a = [] \\
            
            for i in range(0, l): \\
                a.append([]) \\
                for j in range(0, l): \\
                    a[i].append((i * j) \% n) \\
            
            \# get inverse \\
            \# the inverse of a number x is the number y such that (x * y) \% n = 1 \\
            
            \# search s \\
            s = 2 \\
            
           \# search for the inverse of s \\
            for i in range(0, l): \\
                if a[s][i] == 1: \\
                    print(f"Inverse of {s} mod {n} is: {i}", end="") \\
                    break \\
-------------------------------------------------------------------------- \\
        
        $$ 864^{-1} = (450)^5 * (300)^3 $$
        $$ = 150^5 * 3^5 * 300^3 = 150^5 * 3^2 * 3^3 * 300^3 = 150^5 * 3^2 * (3 * 300)^3 $$
        Because we are modding by $899$ we can replace $(3 * 300)^3$ with $1$

        $$ = 150^5 * 3^2 = 150 * 150 * 150 * 150 * 3 * 150 * 3 = 150^3 * (150*3)^2 $$
        $$ = 150^3 * 450^2 = (75 * 2)^3 * 450^2 = 75^3 * 2^3 * 450^2 = 75^3 * 2*2*2 * 450*450 = 75^3 * 2 *900*900 $$
        We can replace those $900$s with $1$ because we are modding by $899$
        $$ 75^3 * 2 = 75^2 * 150 = (25 * 3)^2 * 150 = 25^2 * 3^2 * 150 = 25^2 * 1350 $$
        $$ = 25^2 * (36^2 + 54) = 25^2 * 36^2 + 25^2 * 54 = (25*36)^2 + 25^2 * 54 = 900^2 + 25^2 * 54 $$
        We can the $900^2$ with $1$
        $$ = 1 + 25^2 * 54 = 1 + 625 * 27 * 2 = 1 + (175 + 450) * 27 * 2 = 1 + (2*175 + 900) * 27 $$
        $$ = 1 + 2*175*27 + 900 * 27 = 1 + 2 * 175 * 27 + 27 = 1 + 350 * 27 + 27 = 1 + 27 * (350 + 1) $$
        $$ = 1 + 27 * (351) = 1 + 3 * 9 * (3 * 100 + 51) = 1 + 3 * (3 * 900 + 9 * 51) = 1 + 3 * (3 + 9 * 51) $$
        $$ 1 + 3 * 3 + 3 * 9 * 51 = 1 + 9 + 1377 = 10 + 900 + 477 = 10 + 1 + 477 = 488 $$

        So $x = 488$

        


        



        \newpage	
		\item Story: I took a class on Ancient Greek Philosophy in college, and in the midst of one lecture the professor (extemporaneously) walked the class through Euclid's proof that $\primes$ is infinite.  
		Kudos to the prof for this, since their specialty had nothing to do with Math.  
		But they said ``... the proof also give a way to construct primes''.  
		I didn't bother to correct this, because I'm not one of those students who do that \smiley
		
		Define the {\bf Euclid number} $e_n$ recursively (inspired by the proof of Theorem 13.4) as follows: 
		\begin{eqnarray*}
			e_n & = & e_0e_1e_2 \cdots e_{n-1} + 1 \;\;\;\;\;\; \mbox{ for $n \geq 1$}\\
			e_0 & = & 1
		\end{eqnarray*}
		
		\begin{enumerate} 
			\item Are the Euclid numbers prime?  

                $$ e_0 = 1 $$
                $$ e_1 = e_0 + 1 = 1 + 1 = 2 $$
                $$ e_2 = e_0 e_1 + 1 = 1(2) + 1 = 3 $$
                $$ e_3 = e_0 e_1 e_2 + 1 = 1(2)(3) + 1 = 7 $$
                $$ e_4 = e_0 e_1 e_2 e_3 + 1 = 1(2)(3)(7) + 1 = 43 $$
                $$ e_5 = e_0 e_1 e_2 e_3 e_4 + 1 = 1(2)(3)(7)(43) + 1 = 1807 $$ 
                $$ 1807 = 13 (139) $$

                Thus, all are not prime, but hey some of them are. 

                
			\item Show that $e_m \perp e_n$ for $m \neq n$.

            If $e_m \perp e_n$ then $gcd(a, b) = 1$

            Because of, linear algebra $e_m \perp e_n$ implies $e_n x + e_m y = 1$ for some integers $x$ and $y$.

            If you have $n>m$ then

            $$ e_n = e_0e_1e_2 \cdots e_{n-1} + 1 $$

            $e_m$ would have to be in the product, $e_0e_1e_2 \cdots e_{n-1}$ so some number of times $e_m$ is $e_n$.

            $$ e_n = y e_m + 1 $$

            $$ e_n - y e_m = 1 $$

            Well, there is that linear algebra relationship.

            So then you can infer that $e_m \perp e_n$

            A similar argument would follow for if $n<m$
            
            
                
                
		\end{enumerate}
	\end{enumerate}
	
	\noindent \underline{\hspace{3in}}\\
	
	
	
	
\end{document}