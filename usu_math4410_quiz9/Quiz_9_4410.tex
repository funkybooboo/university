\documentclass[10pt, AMS Euler]{article}
\textheight=9.25in \textwidth=7in \topmargin=-.75in
\oddsidemargin=-0.25in
\evensidemargin=-0.25in
\usepackage{url}  % The bib file uses this
\usepackage{graphicx} %to import pictures
\usepackage{amsmath, amssymb, color, wasysym}
\usepackage{theorem, concrete, multicol,tikz}
\usepackage[normalem]{ulem} %for strikethrough (\sout{blah})

\setlength{\intextsep}{5mm} \setlength{\textfloatsep}{5mm}
\setlength{\floatsep}{5mm}


{\theorembodyfont{\rmfamily}
	\newtheorem{definition}{Definition}[section]}
{\theorembodyfont{\rmfamily} \newtheorem{example}{Example}[section]}
{\theorembodyfont{\rmfamily} \newtheorem{lemma}{Lemma}[section]}
{\theorembodyfont{\rmfamily} \newtheorem{theorem}{Theorem}[section]}
{\theorembodyfont{\rmfamily} \newenvironment{proof}{\par{\it
			Proof:}}{\nopagebreak[4]\rule{2mm}{2jkl;mm}}}
{\theorembodyfont{\rmfamily}
	\newenvironment{solution}{\par{\bf{Solution:}}}{\nopagebreak[4]\rule{2mm}{2mm}}}

\usetikzlibrary{arrows}
\usetikzlibrary{shapes}
\newcommand{\mymk}[1]{%
	\tikz[baseline=(char.base)]\node[anchor=south west, draw,rectangle, rounded corners, inner sep=2pt, minimum size=7mm,
	text height=2mm](char){\ensuremath{#1}} ;}

\newcommand*\circled[1]{\tikz[baseline=(char.base)]{
		\node[shape=circle,draw,inner sep=2pt] (char) {#1};}}


%%%%  SHORTCUT COMMANDS  %%%%
\newcommand{\ds}{\displaystyle}
\newcommand{\Z}{\mathbb{Z}}
\newcommand{\arc}{\rightarrow}
\newcommand{\R}{\mathbb{R}}
\newcommand{\N}{\mathbb{N}}
\newcommand{\Q}{\mathbb{Q}}
\newcommand{\primes}{\mathscr{P}}


%%%%  footnote style %%%%

\renewcommand{\thefootnote}{\fnsymbol{footnote}}

\pagestyle{empty}
\begin{document}
	
	\noindent{\bf \large MATH 4410 }\\
    \noindent{\bf \large Nate Stott A02386053 }\\
	
	\noindent \underline{\hspace{2in}}\\
	
	\noindent {\bf Quiz \#9; Due 11:59 pm, 4/3/2024}\\
	
	
	
	
	\begin{enumerate}

        
		\newpage
		\item Prove that if $2^N +1$ is prime, then $N$ must be a power of $2$.

            Let $p$ be a prime
            $$ 2^N + 1 = p $$
            $$ 2^N = p - 1 $$
            $$ N = \log_2{(p - 1)} $$

            Therefore, if $2^N +1$ is prime, then N must be a power of $2$ (we only care about the $p-1$'s that give us a whole number and the only p-1's that give us a whole number are powers of $2$! (note that is an explanation-point, not a factorial.)).


        \newpage
		\item Let $f_n$ denote the $n^{\mbox{th}}$ \emph{Fermat Number}, defined by $f_n = 2^{2^n}+1$.  
		\begin{enumerate}
			\item Prove that if $n < m$, $f_n \perp f_m$.

                $$ f_m = 2^{2^m}+1 $$
                $$ f_n = 2^{2^n}+1 $$ \\

                $$ f_m = 2^{2^m}+1 $$
                Then n can be $m-1, m-2, ... 0$ \\

                Let $m$ be $3$
                $$ f_3 = 2^{2^3}+1 = 2^8 + 1 = 257 $$
                $$ f_{3-1} = 2^{2^{3-1}}+1 = 2^4 + 1 = 17 $$
                $$ f_{3-2} = 2^{2^{3-2}}+1 = 2^2 + 1 = 5 $$
                $$ f_{3-3} = 2^{2^{3-3}}+1 = 2^1 + 1 = 3 $$ \\

                So I need to look at these guys $2^8$, $2^4$, $2^2$, $2^1$ \\

                Well, none of them are relatively prime to each other because they are powers of 2. But when you add a 1 then the numbers became relatively prime. Let's investigate this. \\

                $3$ is prime \\
                $5$ is prime \\
                $17$ is prime \\
                $257$ is prime \\
                $65537$ is prime \\
                $4294967297$ is not prime since $641 * 6700417$ but those numbers are prime. \\

                So all the Fermat numbers below $f_m$ have unique prime factorizations \\

                Therefore, if $n < m$, $f_n \perp f_m$ \\

			\item Prove that $f_n - 2 = f_0\cdot f_1 \cdot f_2 \cdots f_{n-1}$, for $n \geq 1$ and  $f_0 = 3$.

                $$ f_n = 2^{2^n}+1 $$
                $$ f_0 = 2^{1} + 1 = 3 $$
                $$ f_1 = 2^{2} + 1 = 5 $$
                $$ f_2 = 2^{2^2} + 1 = 2^4 + 1 = 17 $$
                $$ f_3 = 2^{2^3} + 1 = 2^8 + 1 = 257 $$
                $$ f_4 = 2^{2^4} + 1 = 2^{16} + 1 = 65537 $$
                $$ f_5 = 2^{2^5} + 1 = 2^{32} + 1 = 4294967297 $$

                Ok, clearly I can see the pattern
                $$ 3 + 2 = 5 $$
                $$ (3)(5) + 2 = 17 $$
                $$ (3)(5)(17) + 2 = 257 $$

                So from this limited data the recursion is true. But why does the recursion hold true for all $n$?

                Let's try to make sense of why this works. \\
                
                Look at this here table
            \begin{center}
                \begin{tabular}{l|ccccccccc}
                    {\bf Index $n$ } & 0 & 1 & 2 & 3 & 4 & 5 & 6 & 7  \\
                    \hline
                    {Value $2^n$} & 1 & 2 & 4 & 8 & 16 & 32 & 64 & 128   \\
                \end{tabular}
            \end{center}

            $n$ determines the power of $2$ on $2$. Example $2^{2^n}$. \\

            Gathering some data on $2^{2^n}$
            $$ 2^{2^n} $$
            $$ 2^{2^0} = 2^1 = 2 $$
            $$ 2^{2^1} = 2^2 = 4 $$
            $$ 2^{2^2} = 2^4 = 16 $$
            $$ 2^{2^3} = 2^8 = 256 $$
            $$ 2^{2^4} = 2^{16} = 65536 $$
            $$ 2^{2^5} = 2^{32} = 4294967296 $$

            Alright, so adding $1$ gives us the Fermat Numbers. I assume he added $1$ to make the number odd. I also assume he picked, $2^{2^n}$ as it follows the general trend of prime numbers. \\

            Gathering some data about the proposed recurrence.
            $$ f_{0} - 2 = 1 $$
            $$ f_{1} - 2 = (2^{2^0}+1) = (2^1 + 1) = (3) = 3 $$
            $$ f_{2} - 2 = (2^{2^0}+1)(2^{2^1}+1) = (2^1 + 1)(2^2 + 1) = (3)(5) = 15 $$
            $$ f_{3} - 2 = (2^{2^0}+1)(2^{2^1}+1)(2^{2^2}+1) = (2^1 + 1)(2^2 + 1)(2^4 + 1) = (3)(5)(17) = 255 $$
            $$ f_{4} - 2 = (2^{2^0}+1)(2^{2^1}+1)(2^{2^2}+1)(2^{2^3}+1) = (2^1 + 1)(2^2 + 1)(2^4 + 1)(2^8 + 1) = (3)(5)(17)(257) = 65535 $$
            $$ f_{4} - 2 = (2^{2^0}+1)(2^{2^1}+1)(2^{2^2}+1)(2^{2^3}+1)(2^{2^4}+1) = (3)(5)(17)(257)(65537) = 4294967295 $$
            Why do you always have to add $2$ to get to the Fermat Number? \\

            $$ 5 - 2 = 2^1 + 1 $$

            $$ 17 - 2 = (2^1 + 1)(2^2 + 1) $$
            $$ 17 - 2 = 2^1 * 2^2 + 2^1 + 2^2 + 1 $$
            $$ 17 - 2 = 2^{1+2} + 2^1 + 2^2 + 1 $$
            $$ 17 - 2 = 2^3 + 2^1 + 2^2 + 1 $$
            $$ 17 - 2 = 2^3 + 2^2 + 2^1 + 1 $$

            $$ 257 - 2 = (2^1 + 1)(2^2 + 1)(2^4 + 1) $$
            $$ 257 - 2 = (2^3 + 2^2 + 2^1 + 1)(2^4 + 1) $$
            $$ 257 - 2 = 2^3*2^4 + 2^3 + 2^2*2^4 + 2^2 + 2^1*2^4 + 2^1 + 2^4 + 1 $$
            $$ 257 - 2 = 2^{3+4} + 2^3 + 2^{2+4} + 2^2 + 2^{1+4} + 2^1 + 2^4 + 1 $$
            $$ 257 - 2 = 2^7 + 2^3 + 2^6 + 2^2 + 2^5 + 2^1 + 2^4 + 1 $$
            $$ 257 - 2 = 2^7 + 2^6 + 2^5 + 2^4 + 2^3 + 2^2 + 2^1 + 1 $$

            Alright, clearly a summation is at work. \\
            
            Let $N=2^{n}$
            $$ f_n = 2^{N} + 1 = (\sum_{k=0}^{N-1} 2^k) + 2 = f_0\cdot f_1 \cdot f_2 \cdots f_{n-1} + 2 $$
            
                
		\end{enumerate}
		
		\newpage
		\item Compute $199^{65571}\!\! \mod 193$.  Do as much of this as possible with your brain, and show your work.

        Fact 1
        $$ a^{p-1} \equiv 1 \mod p $$

        $$ 199^{65571} \mod 193 $$
        mod the $199$
        $$ 6^{65571} \mod 193 $$

        $192$ goes into $65571$ → $341$ with a remainder of $99$

        $$ 6^{(192)(341)+99} \mod 193 $$
        I can get rid of $192$ in the exponent because of Fact 1

        $$ 6^{99} \mod 193 $$
        $6*6*6 \mod 193$ is $23$ and we are doing that $33$ times
        $$ 23^{33} \mod 193 $$

        $$ 23^{32} * 23 \mod 193 $$
        I did the last bit with a calculator but $23^{32} \mod 193$ is $1$

        $$ 23 \mod 193 $$



        \newpage
		\item Find the smallest positive integer $x$ such that 
		\begin{align*}
			x &\equiv 11 \!\!\! \mod 24,  \mbox{ and}\\ 
			x &\equiv 16 \!\!\! \mod 35.
		\end{align*}

            $$ x = 16 + 35 q $$
            $$ x = 11 + 24 k $$
            Subtract equations
            $$ 0 = 5 + 35q +24(-k) $$
            $$ -5 = 35q +24(-k) $$
            $$ 5 = 35(-q) +24k $$

            Do the shifty-shifty algorithm.

            $$ 35 = 24(1) + 11 -> 11 = -24(1)+35 $$
            $$ 24 = 11(2) + 2 -> 2 = -11(2) + 24 $$
            $$ 11 = 2(5) + 1 -> 1 = -2(5) + 11 $$

            Do the continuation-of-the-shifty-shifty algorithm.

            $$ 1 = -2(5) + 11 $$
            Plug in $-11(2) + 24$ for $2$
            $$ 1 = -(-11(2) + 24)(5) + 11 $$
            $$ 1 = 11(10) - 24(5) + 11 $$
            $$ 1 = 11(11) - 24(5) $$
            Plug in $-24(1)+35$ for $11$
            $$ 1 = (-24(1)+35)(11) - 24(5) $$
            $$ 1 = -24(11)+35(11) - 24(5) $$
            $$ 1 = -24(16) + 35(11) $$
            Now I need to multiply both sides by 5 to make it look like the third equation from the top.
            $$ 5 = 24(-16)(5) + 35(11)(5) $$
            
            $$ 5 = 35(11)(5) + 24(-16)(5) $$
            $$ 5 = 35(-q) +24k $$
            Alright so 
            $$ -q = (11)(5) $$
            $$ q = (-11)(5) $$
            $$ k = (-16)(5) $$
            Plugging into the originals
            $$ x = 16 + 35 q $$
            $$ x = 16 + 35 (-11)(5) $$
            $$ x = -1909 $$
            But remember this is being modded by 840
            $$ x = 611 $$
            \\
            $$ x = 11 + 24 k $$
            $$ x = 11 + 24 (-16)(5) $$
            $$ x = -1909 $$
            
            Alright so let's check the work

            $$ 611 mod 35 = 16 $$
            $$ 611 mod 24 = 11 $$

            Well, that looks right to me!
            

  
		\newpage
		\item  Determine (and display) the prime factorization of $50!$.  (Note that the sentence is not an exclamation! You're asked to find the prime factorization of $50$-factorial.) 

        How many numbers from [1..50] are divisible by $2$? Well 25, all the even numbers.
        How about $4$? Well 12, it's half of the even numbers are also divisible by 4. 

        Here is table displaying this information

            \begin{center}
                \begin{tabular}{l|cccccc}
                    {$n$ } & $n$ & $n^2$ & $n^3$ & $n^4$ & $n^5$  \\
                    \hline
                    {$2$ } & 25 & 12 & 6 & 3 & 1  \\
                    {$3$} & 18 & 5 & 1 & &   \\
                    {$5$} & 10 & 2 & & &   \\
                    {$7$} & 7 & 1 & & &   \\
                    {$11$} & 4 & & & &   \\
                    {$13$} & 3 & & & &   \\
                    {$17$} & 2 & & & &   \\
                    {$19$} & 2 & & & &   \\
                    {$23$} & 2 & & & &   \\
                    {$29$} & 1 & & & &   \\
                    {$31$} & 1 & & & &   \\
                    {$37$} & 1 & & & &   \\
                    {$41$} & 1 & & & &   \\
                    {$43$} & 1 & & & &   \\
                    {$47$} & 1 & & & &   \\
                    
                \end{tabular}
            \end{center}

            So the prime factorization of $50!$ is 
            $$ 50! = 2^{25+12+6+3+1} * 3^{18+5+1} * 5^{10+2} * 7^{7+1} * 11^{4} * 13^{3} * 17^{2} * 19^{2} * 23^{2} * 29^{1} * 31^{1} * 37^{1} * 41^{1} * 43^{1} * 47^{1} $$
  
	\end{enumerate}
	
	\noindent \underline{\hspace{3in}}\\
	
	
	
	
\end{document}